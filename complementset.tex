

\documentclass[11pt,a4paper,titlepage,oneside,openany]{article}

\pagestyle{plain}
%\renewcommand{\baselinestretch}{1.7}

\usepackage{setspace}
%\singlespacing
\onehalfspacing
%\doublespacing
%\setstretch{1.1}

\usepackage{amsmath}
\usepackage{amssymb}
\usepackage{amsthm}
\usepackage{framed}
\usepackage{multicol}

\usepackage[margin=3cm]{geometry}
\usepackage{graphicx,psfrag}%\usepackage{hyperref}
\usepackage[small]{caption}
\usepackage{subfig}

\usepackage{algorithm}
\usepackage{algorithmic}
\newcommand{\theHalgorithm}{\arabic{algorithm}}

\usepackage{varioref} %NB: FIGURE LABELS MUST ALWAYS COME DIRECTLY AFTER CAPTION!!!
%\newcommand{\vref}{\ref}

\usepackage{index}
\makeindex
\newindex{sym}{adx}{and}{Symbol Index}
%\newcommand{\symindex}{\index[sym]}
%\newcommand{\symindex}[1]{\index[sym]{#1}\hfill}
\newcommand{\symindex}[1]{\index[sym]{#1}}

%\usepackage[breaklinks,dvips]{hyperref}%Always put after varioref, or you'll get nested section headings
%Make sure this is after index package too!
%\hypersetup{colorlinks=false,breaklinks=true}
%\hypersetup{colorlinks=false,breaklinks=true,pdfborder={0 0 0.15}}


%\usepackage{breakurl}

\graphicspath{{./images/}}

\usepackage[subfigure]{tocloft}%For table of contents
\setlength{\cftfignumwidth}{3em}

\input{longdiv}
\usepackage{wrapfig}


%\usepackage{index}
%\makeindex
%\usepackage{makeidx}

%\usepackage{lscape}
\usepackage{pdflscape}
\usepackage{multicol}

\usepackage[utf8]{inputenc}

%\usepackage{fullpage}

%Compulsory packages for the PhD in UL:
%\usepackage{UL Thesis}
\usepackage{natbib}

%\numberwithin{equation}{section}
\numberwithin{equation}{section}
\numberwithin{algorithm}{section}
\numberwithin{figure}{section}
\numberwithin{table}{section}
%\newcommand{\vec}[1]{\ensuremath{\math{#1}}}

%\linespread{1.6} %for double line spacing

\usepackage{afterpage}%fingers crossed

\newtheorem{thm}{Theorem}[section]
\newtheorem{defin}{Definition}[section]
\newtheorem{cor}[thm]{Corollary}
\newtheorem{lem}[thm]{Lemma}

%\newcommand{\dbar}{{\mkern+3mu\mathchar'26\mkern-12mu d}}
\newcommand{\dbar}{{\mkern+3mu\mathchar'26\mkern-12mud}}

\newcommand{\bbSigma}{{\mkern+8mu\mathsf{\Sigma}\mkern-9mu{\Sigma}}}
\newcommand{\thrfor}{{\Rightarrow}}

\newcommand{\mb}{\mathbb}
\newcommand{\bx}{\vec{x}}
\newcommand{\bxi}{\boldsymbol{\xi}}
\newcommand{\bdeta}{\boldsymbol{\eta}}
\newcommand{\bldeta}{\boldsymbol{\eta}}
\newcommand{\bgamma}{\boldsymbol{\gamma}}
\newcommand{\bTheta}{\boldsymbol{\Theta}}
\newcommand{\balpha}{\boldsymbol{\alpha}}
\newcommand{\bmu}{\boldsymbol{\mu}}
\newcommand{\bnu}{\boldsymbol{\nu}}
\newcommand{\bsigma}{\boldsymbol{\sigma}}
\newcommand{\bdiff}{\boldsymbol{\partial}}

\newcommand{\tomega}{\widetilde{\omega}}
\newcommand{\tbdeta}{\widetilde{\bdeta}}
\newcommand{\tbxi}{\widetilde{\bxi}}



\newcommand{\wv}{\vec{w}}

\newcommand{\ie}{i.e. }
\newcommand{\eg}{e.g. }
\newcommand{\etc}{etc}

\newcommand{\viceversa}{vice versa}
\newcommand{\FT}{\mathcal{F}}
\newcommand{\IFT}{\mathcal{F}^{-1}}
%\renewcommand{\vec}[1]{\boldsymbol{#1}}
\renewcommand{\vec}[1]{\mathbf{#1}}
\newcommand{\anged}[1]{\langle #1 \rangle}
\newcommand{\grv}[1]{\grave{#1}}
\newcommand{\asinh}{\sinh^{-1}}

\newcommand{\sgn}{\text{sgn}}
\newcommand{\morm}[1]{|\det #1 |}

\newcommand{\galpha}{\grv{\alpha}}
\newcommand{\gbeta}{\grv{\beta}}
%\newcommand{\rnlessO}{\mb{R}^n \setminus \vec{0}}
\usepackage{listings}

\interfootnotelinepenalty=10000

\newcommand{\sectionline}{%
  \nointerlineskip \vspace{\baselineskip}%
  \hspace{\fill}\rule{0.5\linewidth}{.7pt}\hspace{\fill}%
  \par\nointerlineskip \vspace{\baselineskip}
}

\renewcommand{\labelenumii}{\roman{enumii})}


\documentclass[12pt]{article}
%\usepackage[final]{pdfpages}

\usepackage{graphicx}
\graphicspath{{/Users/kevinhayes/Documents/teaching/images/}}

\usepackage{tikz}
\usetikzlibrary{arrows}

\newcommand{\bbr}{\Bbb{R}}
\newcommand{\zn}{\Bbb{Z}^n}

%\usepackage{epsfig}
%\usepackage{subfigure}
\usepackage{amscd}
\usepackage{amssymb}
\usepackage{amsbsy}
\usepackage{amsthm}
\usepackage{natbib}
\usepackage{amsbsy}
\usepackage{enumerate}
\usepackage{amsmath}
\usepackage{eurosym}
%\usepackage{beamerarticle}
\usepackage{txfonts}
\usepackage{fancyvrb}
\usepackage{fancyhdr}
\usepackage{natbib}
\bibliographystyle{chicago}

\usepackage{vmargin}
% left top textwidth textheight headheight
% headsep footheight footskip
\setmargins{2.0cm}{2.5cm}{16 cm}{22cm}{0.5cm}{0cm}{1cm}{1cm}
\renewcommand{\baselinestretch}{1.3}


\pagenumbering{arabic}
\begin{document}

%------------------------------------------------------%

\section*{Complement of a Set}
%(2.3.1) 
Consider the universal set $U$ such that
\[U=\{2,4,6,8,10,12,15\} \]
For each of the sets $A$,$B$,$C$ and $D$, specify the complement sets.
{

\begin{center}
\begin{tabular}{|c|c|}
\hline
Set &\phantom{sp} Complement \phantom{sp}\\
\hline \phantom{sp} $A=\{4,6,12,15\}$ \phantom{sp} &
$A^{\prime}=\{2,8,10\}$ \\ \hline $B=\{4,8,10,15\}$ & \\ \hline
$C=\{2,6,12,15\}$ & \\ \hline $D=\{8,10,15\}$ & \\ \hline

\end{tabular}
\end{center}
}




\section{Differences and complements}


\begin{itemize}
\item If A and B are sets then the difference set A − B is the set of all elements
of A which do not belong to B.
\item If B is a sub-set of A, then A − B is sometimes called the complement of
B in A. When A is the universal set one may simply refer to the
complement of B to denote all things not in B. \item The complement of a set A
is denoted as $A^c$ or $A^{c}$.
\end{itemize}



%--------------------------------------------------------%
\subsubsection*{Complements}
\begin{itemize}

\item The Complements of A and B are the elements of the universal set not contained in A and B.

\item The complements are denoted $\boldsymbol{A}^{c}$ and $\boldsymbol{B}^{c}$
\[ \boldsymbol{A}^{c} = \{4,6,8,9\}, \]
\[ \boldsymbol{B}^{c} = \{1,3,5,7,9\}, \]
\end{itemize}


%=========================================%
\subsection{Complement of a Set}
The complement of a set A (denoted by $A^{c}$) is the set of elements which are not in set A. Hence, $A^{c}={x|x \notin A}$.
More specifically, $A^{c}=(U-A)$ where $U$ is a universal set which contains all objects.

\subsection{Example} - If A={x|x belongstosetofoddintegers}A={x|x belongstosetofoddintegers} then $A^{c}={y|y \mbox{does not belong toset ofodd integers}}$

Complement Set






\section*{Complement of a Set}
%(2.3.1) 
Consider the universal set $U$ such that
\[U=\{2,4,6,8,10,12,15\} \]
For each of the sets $A$,$B$,$C$ and $D$, specify the complement sets.
{

\begin{center}
\begin{tabular}{|c|c|}
\hline
Set &\phantom{sp} Complement \phantom{sp}\\
\hline \phantom{sp} $A=\{4,6,12,15\}$ \phantom{sp} &
$A^{c}=\{2,8,10\}$ \\ \hline $B=\{4,8,10,15\}$ & \\ \hline
$C=\{2,6,12,15\}$ & \\ \hline $D=\{8,10,15\}$ & \\ \hline

\end{tabular}
\end{center}
}

\end{document}


 


