\documentclass[11pt,a4paper,titlepage,oneside,openany]{article}

\pagestyle{plain}
%\renewcommand{\baselinestretch}{1.7}

\usepackage{setspace}
%\singlespacing
\onehalfspacing
%\doublespacing
%\setstretch{1.1}

\usepackage{amsmath}
\usepackage{amssymb}
\usepackage{amsthm}
\usepackage{framed}
\usepackage{multicol}

\usepackage[margin=3cm]{geometry}
\usepackage{graphicx,psfrag}%\usepackage{hyperref}
\usepackage[small]{caption}
\usepackage{subfig}

\usepackage{algorithm}
\usepackage{algorithmic}
\newcommand{\theHalgorithm}{\arabic{algorithm}}

\usepackage{varioref} %NB: FIGURE LABELS MUST ALWAYS COME DIRECTLY AFTER CAPTION!!!
%\newcommand{\vref}{\ref}

\usepackage{index}
\makeindex
\newindex{sym}{adx}{and}{Symbol Index}
%\newcommand{\symindex}{\index[sym]}
%\newcommand{\symindex}[1]{\index[sym]{#1}\hfill}
\newcommand{\symindex}[1]{\index[sym]{#1}}

%\usepackage[breaklinks,dvips]{hyperref}%Always put after varioref, or you'll get nested section headings
%Make sure this is after index package too!
%\hypersetup{colorlinks=false,breaklinks=true}
%\hypersetup{colorlinks=false,breaklinks=true,pdfborder={0 0 0.15}}


%\usepackage{breakurl}

\graphicspath{{./images/}}

\usepackage[subfigure]{tocloft}%For table of contents
\setlength{\cftfignumwidth}{3em}

\input{longdiv}
\usepackage{wrapfig}


%\usepackage{index}
%\makeindex
%\usepackage{makeidx}

%\usepackage{lscape}
\usepackage{pdflscape}
\usepackage{multicol}

\usepackage[utf8]{inputenc}

%\usepackage{fullpage}

%Compulsory packages for the PhD in UL:
%\usepackage{UL Thesis}
\usepackage{natbib}

%\numberwithin{equation}{section}
\numberwithin{equation}{section}
\numberwithin{algorithm}{section}
\numberwithin{figure}{section}
\numberwithin{table}{section}
%\newcommand{\vec}[1]{\ensuremath{\math{#1}}}

%\linespread{1.6} %for double line spacing

\usepackage{afterpage}%fingers crossed

\newtheorem{thm}{Theorem}[section]
\newtheorem{defin}{Definition}[section]
\newtheorem{cor}[thm]{Corollary}
\newtheorem{lem}[thm]{Lemma}

%\newcommand{\dbar}{{\mkern+3mu\mathchar'26\mkern-12mu d}}
\newcommand{\dbar}{{\mkern+3mu\mathchar'26\mkern-12mud}}

\newcommand{\bbSigma}{{\mkern+8mu\mathsf{\Sigma}\mkern-9mu{\Sigma}}}
\newcommand{\thrfor}{{\Rightarrow}}

\newcommand{\mb}{\mathbb}
\newcommand{\bx}{\vec{x}}
\newcommand{\bxi}{\boldsymbol{\xi}}
\newcommand{\bdeta}{\boldsymbol{\eta}}
\newcommand{\bldeta}{\boldsymbol{\eta}}
\newcommand{\bgamma}{\boldsymbol{\gamma}}
\newcommand{\bTheta}{\boldsymbol{\Theta}}
\newcommand{\balpha}{\boldsymbol{\alpha}}
\newcommand{\bmu}{\boldsymbol{\mu}}
\newcommand{\bnu}{\boldsymbol{\nu}}
\newcommand{\bsigma}{\boldsymbol{\sigma}}
\newcommand{\bdiff}{\boldsymbol{\partial}}

\newcommand{\tomega}{\widetilde{\omega}}
\newcommand{\tbdeta}{\widetilde{\bdeta}}
\newcommand{\tbxi}{\widetilde{\bxi}}



\newcommand{\wv}{\vec{w}}

\newcommand{\ie}{i.e. }
\newcommand{\eg}{e.g. }
\newcommand{\etc}{etc}

\newcommand{\viceversa}{vice versa}
\newcommand{\FT}{\mathcal{F}}
\newcommand{\IFT}{\mathcal{F}^{-1}}
%\renewcommand{\vec}[1]{\boldsymbol{#1}}
\renewcommand{\vec}[1]{\mathbf{#1}}
\newcommand{\anged}[1]{\langle #1 \rangle}
\newcommand{\grv}[1]{\grave{#1}}
\newcommand{\asinh}{\sinh^{-1}}

\newcommand{\sgn}{\text{sgn}}
\newcommand{\morm}[1]{|\det #1 |}

\newcommand{\galpha}{\grv{\alpha}}
\newcommand{\gbeta}{\grv{\beta}}
%\newcommand{\rnlessO}{\mb{R}^n \setminus \vec{0}}
\usepackage{listings}

\interfootnotelinepenalty=10000

\newcommand{\sectionline}{%
  \nointerlineskip \vspace{\baselineskip}%
  \hspace{\fill}\rule{0.5\linewidth}{.7pt}\hspace{\fill}%
  \par\nointerlineskip \vspace{\baselineskip}
}

\renewcommand{\labelenumii}{\roman{enumii})}

\begin{document}


\section*{Venn Diagrams}

Draw a Venn Diagram to represent the universal set
$\mathcal{U} = \{0,1,2,3,4,5,6\}$ with subsets:\\
$A = \{2,4,5\}$\\
$B = \{1,4,5,6\}$\\

\noindent Find each of the following
\begin{itemize}
\item[(a)] $A \cup B $
\item[(b)] $A \cap B $
\item[(c)] $A-B$
\item[(d)] $B-A$
\item[(e)] $A \oplus B$
\end{itemize}
\newpage



76 Management mathematics
%------------------------------------------------------------------------- %

\section{1.7 Venn diagrams}
Often the relationships that exist between sets can best be shown using a
Venn diagram. To construct a Venn diagram we let a certain region, usually
a rectangle, represent the universal set.

This rectangle is often implied
by the constraints of the page and only in those circumstances where its
boundary is important is the rectangle drawn (see the diagrams below for
example). Individual sets are then represented by regions, often circles,
within this rectangle. One can then easily depict intersections, unions,
complements, etc. on the diagram. For example:
%==================================================%

\begin{itemize}
\item Let $A$,$B$ and $C$ be subsets of a universal set $U$.
\item Draw a labelled Venn diagram depicting $A$,$B$,$C$ in such a way that they divide $U$ into 8 disjoint regions.
\end{itemize}
\begin{center}
\begin{tabular}{|c|c|c|c|c|}
\hline Region & A & B & C &  \\
\hline 1 &\phantom{sp} No \phantom{sp}&\phantom{sp} No\phantom{sp} & \phantom{sp} No \phantom{sp} &  \\
\hline 2 & Yes & No & No &  \\
\hline 3 & No & Yes & No &  \\
\hline 4 & Yes & Yes & No &  \\
\hline 5 & No & No & Yes &  \\
\hline 6 & Yes & No & Yes &  \\
\hline 7 & No & Yes & Yes &  \\
\hline 8 & Yes & Yes & Yes & $A \cap B \cap C$ \\
\hline
\end{tabular}
\end{center}

\begin{center}
\begin{tabular}{|c|c|c|c|c|}
\hline Region & A & B & C &  \\
\hline 1 &\phantom{sp} No \phantom{sp}&\phantom{sp} No\phantom{sp} & \phantom{sp} No \phantom{sp} &  \\
\hline 2 & Yes & No & No &  \\
\hline 3 & No & Yes & No &  \\
\hline 4 & Yes & Yes & No &  \\
\hline 5 & No & No & Yes &  \\
\hline 6 & Yes & No & Yes &  \\
\hline 7 & No & Yes & Yes &  \\
\hline 8 & Yes & Yes & Yes & $A \cap B \cap C$ \\
\hline
\end{tabular}
\end{center}

\begin{center}
\begin{tabular}{|c|c|c|c|c|}
\hline Region & A & B & C &  \\
\hline 1 &\phantom{s} No \phantom{s}&\phantom{s} No\phantom{s} & \phantom{s} No \phantom{s} & $(A \cup B \cup C)^C$  \\
\hline 2 & Yes & No & No &  \\
\hline 3 & No & Yes & No &  $B-(A \cap C)$\\
\hline 4 & Yes & Yes & No & $(A \cap B) - C$ \\
\hline 5 & No & No & Yes &  $C-(A \cap B)$ \\
\hline 6 & Yes & No & Yes & $(A \cap C) - B$ \\
\hline 7 & No & Yes & Yes & $(B \cap C) - A$ \\
\hline 8 & Yes & Yes & Yes & $A \cap B \cap C$ \\
\hline
\end{tabular}
\end{center}

\end{document}
