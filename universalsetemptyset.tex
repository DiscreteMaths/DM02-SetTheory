
%--------------------------------- %
\subsection{Universal Set and the Empty Set}
The first is the \textbf{\textit{universal set}}, typically denoted $U$. This set is all of the elements that we may choose from. This set may be different from one setting to the next. 

For example one universal set may be the set of real numbers whereas for another problem the universal set may be the whole numbers $\{0, 1, 2,\ldots\}$.

The other set that requires consideration is called the \textit{\textbf{empty set}}. The empty set is the unique set is the set with no elements. We write this as $\{ \}$ and denote this set by $\emptyset$.
%---------------------------------------------------------------------------%
\textbf{Complement and universal set}

The universal set (if it exists), usually denoted U, is a set of which everything conceivable is a member. In pure set theory, normally sets are the only objects considered (unlike here, where we have also considered numbers, colours and books, for example); in this case U would be the set of all sets. (Non-set objects, where they are allowed, are called 'urelemente' and are discussed below.)

In the presence of a universal set we can define X′, the complement of X, to be $U−X$. X′ contains everything in the universe apart from the elements of X. We could alternatively have defined it as

\[X′ = \{x | \tilde (x\in X)\}\]
and U as the complement of the empty set.
