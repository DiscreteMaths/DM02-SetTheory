
\documentclass[12pt]{article}
%\usepackage[final]{pdfpages}

\usepackage{graphicx}
\graphicspath{{/Users/kevinhayes/Documents/teaching/images/}}

\usepackage{tikz}
\usetikzlibrary{arrows}

\newcommand{\bbr}{\Bbb{R}}
\newcommand{\zn}{\Bbb{Z}^n}

%\usepackage{epsfig}
%\usepackage{subfigure}
\usepackage{amscd}
\usepackage{amssymb}
\usepackage{amsbsy}
\usepackage{amsthm}
\usepackage{natbib}
\usepackage{amsbsy}
\usepackage{enumerate}
\usepackage{amsmath}
\usepackage{eurosym}
%\usepackage{beamerarticle}
\usepackage{txfonts}
\usepackage{fancyvrb}
\usepackage{fancyhdr}
\usepackage{natbib}
\bibliographystyle{chicago}

\usepackage{vmargin}
% left top textwidth textheight headheight
% headsep footheight footskip
\setmargins{2.0cm}{2.5cm}{16 cm}{22cm}{0.5cm}{0cm}{1cm}{1cm}
\renewcommand{\baselinestretch}{1.3}


\pagenumbering{arabic}

\begin{document}
\section{Discrete Maths - Logic Tutorial Sheet}

%==================================================%

\subsection{Notation - Listing Method}

By convention, a set can be written by \textbf{listing} its elements, separated by commas, between {braces}. 

Using the sets defined above:
\begin{itemize}
\item A = $\{1, 2, 3\}$
\item B = $\{red, green, blue\}$
\item C = $\{\}$
\end{itemize}
This is impractical for large sets (D), and impossible for infinite ones (E). 
\subsection{Notation - Building Method}
Thus a set can also be described by naming a particular property that is shared by all its elements and only by them. A common notation uses a bar (|) to separate a variable name (e.g. "x") from a property of the variable that elements of the set must have. For example:
\begin{itemize}
\item D = $\{x |$ x is a book and x is in the British Library $\}$
\item E = $\{x |$ x is a positive integer$\}$
\end{itemize}
A simple translation of this notation is that "\textit{D is the set of all x, where x is a book and x is in the British Library}" or "\textit{E is the set of all x, where x is a positive integer}".

%======================================================================== %



\section*{Rules of Inclusion, Listing and Cardinality}
For each of the following sets, a set is specified by the rules of inclusion method and listing method respectively. Also stated is the cardinality of that data set.
\subsection*{Worked example 1}
\begin{itemize}
\item $\{ x : x $ is an odd integer $ 5 \leq x \leq 17 \}$
\item $x = \{5,7,9,11,13,15,17\}$
\item The cardinality of set $x$ is 7.
\end{itemize}

\subsection*{Worked example 2}
\begin{itemize}
\item $\{ y : y $ is an even integer $ 6 \leq y < 18 \}$
\item $y = \{6,8,10,12,14,16\}$
\item The cardinality of set $y$ is 6.
\end{itemize}

\subsection*{Worked example 3}
A perfect square is a number that has a integer value as a
square root. 4 and 9 are perfect squares ($\sqrt{4} = 2$,
$\sqrt{9} = 3$).
\begin{itemize}
\item $\{ z : z $ is an perfect square $ 1 < z < 100 \}$
\item $z = \{4,9,16,25,36,49,64,81\}$
\item The cardinality of set $z$ is 8.
\end{itemize}



%===================================================================================== %
\subsection*{Question 1}

\begin{itemize}
\item $\{ s :  \mbox{ s is an odd integer and } 2 \leq s \leq 10 \}$
\item $\{ 2m :  m \in Z \mbox{ and }5 \leq m \leq 10 \}$
\item $\{ 2^t :  t \in Z \mbox{ and } 0 \leq t \leq 5 \}$
\end{itemize}


%\begin{center}
%\begin{tabular}{|c|c|c|c|c|c|}
%\hline n &  &{\color{white} space}  & {\color{white} space} &  &  \\ 
%\hline 2n+1 &{\color{white} space}  &  &  & {\color{white} space} &{\color{white} space}  \\ 
%\hline 
%\end{tabular} 
%\end{center}

\newpage
%=====================================================%
\ssection{Roster or Tabular Form}
The set is represented by listing all the elements comprising it. The elements are enclosed within braces and separated by commas.

\begin{itemize}
\item Example 1 : Set of vowels in English alphabet, $A=\{a,e,i,o,u\}$
\item Example 2 : Set of odd numbers less than 10, $B=\{1,3,5,7,9\}$
\end{itemize}
%====================%
\subsection{Set Builder Notation}
The set is defined by specifying a property that elements of the set have in common. The set is described as A={x:p(x)}A={x:p(x)}
Example 1 : The set {a,e,i,o,u}{a,e,i,o,u} is written as :

\[A={x:x is a vowel in English alphabet}\]


\begin{itemize}
\item Specifying Sets
\item Listing Method
\item Rules of Inclusion method
\end{itemize}
%------------------------------------------------------------------------------------------------------ %
\section{2.1 Specifying sets}
%Learning Objectives
\smallskip 
When you have completed your study of this section, you should be able to:
\begin{itemize}
\item use set notation for specifying sets by the listing method and rules of inclusion method;
\item use and interpret the standard symbols for special sets of numbers and for the empty set.
\end{itemize}

\smallskip 
%---------------------------------------------------- %
\subsection{2.1.1 Listing method}
\smallskip 
We usually use an upper case letter to denote a set and a lower case letter to denote a member of
the set. To specify a set, we must describe its members in an unambiguous way. One way of doing
this is to list the members of the set, separated by commas, and enclose the list in a pair of brace
brackets.
\smallskip 
%--------------------------------------- %
\smallskip 
%Example 2.1 
\begin{itemize}
\item The set D of decimal digits can be expressed as
\[D = {0, l,2,3,4,5,6,7,8,9}\]
\item The set B of bits can be expressed as
\[B= {0,1}\]. 
\end{itemize}
\smallskip 

\subsection{2.1.2 Rules of inclusion method}
\smallskip 
\subsection{Rules of Inclusion}
Another way of specifying a set is by giving rules of inclusion that distinguish members of the
set from objects not in the set.

Definition 2.3 The context of the problem in which u set arises determines an underlying set,
called the universal set for the problem, from which the elements of the set will be drawn.
\smallskip 
%----------------------------------------- %
\smallskip 
For example, if our subject is a set of leopards,~the universal set, explicitly stated or implied by the
context, might be all wild animals in Africa or all animals in London Zoo or all animals belonging
to the cat family.

Example 2.4 To specify the set H of Example 2.3, we could write
\[H=\{n \in Z : 1 \leq n \leq 100\} \]

\smallskip 




%------------------------------------------------------------------------%
\begin{itemize}
	\item[(i)] Describe the following set by the listing method
	\[ \{ 2r+1 : r \in Z^{+} and r \leq 5  \} \]
	\item[(ii)] Let A,B be subsets of the universal set U.
	
\end{itemize}

\section{Specifying Sets}

\begin{enumerate}
\item Listing Method
\item Rules of Inclusion
\end{enumerate}


%-------------------------------------- %

\begin{itemize}
\item[\textbf{A}] $ \{ 2n : n \in \mathbb{Z^{+}} \} $
\item[\textbf{B}] $ \{ 3,6,9,12,15,18,\ldots \} $
\end{itemize}
\textbf{Questions}
\begin{itemize}
\item[(i)] \textbf{A} is described by the rules of inclusion. Describe \textbf{A} with the listing method.
\item[(ii)] \textbf{B} is described by the listing method. Describe \textbf{B} with the rules of inclusion. 
\end{itemize}
