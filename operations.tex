
\documentclass[12pt]{article}
%\usepackage[final]{pdfpages}

\usepackage{graphicx}
\graphicspath{{/Users/kevinhayes/Documents/teaching/images/}}

\usepackage{tikz}
\usetikzlibrary{arrows}

\newcommand{\bbr}{\Bbb{R}}
\newcommand{\zn}{\Bbb{Z}^n}

%\usepackage{epsfig}
%\usepackage{subfigure}
\usepackage{amscd}
\usepackage{amssymb}
\usepackage{amsbsy}
\usepackage{amsthm}
\usepackage{multicol}
\usepackage{natbib}
\usepackage{amsbsy}
\usepackage{enumerate}
\usepackage{amsmath}
\usepackage{eurosym}
%\usepackage{beamerarticle}
\usepackage{txfonts}
\usepackage{fancyvrb}
\usepackage{fancyhdr}
\usepackage{natbib}
\bibliographystyle{chicago}

\usepackage{vmargin}
% left top textwidth textheight headheight
% headsep footheight footskip
\setmargins{2.0cm}{2.5cm}{16 cm}{22cm}{0.5cm}{0cm}{1cm}{1cm}
\renewcommand{\baselinestretch}{1.3}


\pagenumbering{arabic}

\begin{document}

%--------------------------------------------------------%
\subsubsection*{Complements}
\begin{itemize}

\item The Complements of A and B are the elements of the universal set not contained in A and B.

\item The complements are denoted $\mathcal{A}^{c}$ and $\mathcal{B}^{c}$
\[ \mathcal{A}^{c} = \{4,6,8,9\}, \]
\[ \mathcal{B}^{c} = \{1,3,5,7,9\}, \]

\end{itemize}


%--------------------------------------------------------%

\subsubsection*{Intersection}
\begin{itemize}

\item Intersection of two sets describes the elements that are members of both the specified Sets

\item The intersection is denoted $\mathcal{A\cap B}$
\[ \mathcal{A\cap B} = \{2\}\]

\item only one element is a member of both A and B.
\end{itemize}
%--------------------------------------------------------%


\subsection*{Union and intersection of sets}

\begin{itemize}
\item The \textbf{union} of two sets A and B is a set containing all the elements in
either A or B (or both)
i.e.
\[A \cup B = {x / x \in A \mbox{ or } x \in B}.\]
\item The \textbf{intersection} of two sets A and B is a set containing all the elements
that are both in A and B
i.e.
\[A \cap B = {x / x \in A \mbox{ and }x \in B}\].

\item If sets A and B have no elements in common, i.e. $A \cap B = \emptyset$,then A and B
are termed \textbf{disjoint sets}.
\end{itemize}


\subsection{Set operations}

Suppose X and Y are sets. Various operations allow us to build new sets from them.

\begin{description}
\item[Union]
The union of X and Y, written $X\cup Y$, contains all the elements in X and all those in Y. Thus $A \cup B = \{1, 2, 3, red, green, blue\}$. 
As A is a subset of E, the set $A \cup E$ is just E.

\item[Intersection]

The intersection of X and Y, written $X \cap Y$, contains all the elements that are common to both X and Y. Thus ${1,2,3,red,green,blue} \cap {2,4,6,8,10} = {2}$.


\end{description}

%---------------------------------%
%------------------------------------------------------------------------%



\section{Set Theory Operations}
\begin{enumerate}
\item The Universal Set $\mathcal{U}$
\item Union
\item Intersection
\item Set Difference
\item Relative Difference
\end{enumerate}

\section*{Set Operations}
\begin{itemize}
\item Union ($\cup$) - also known as the \textbf{OR} operator. A union signifies a bringing together. The union of the sets A and B consists of the elements that are in either A or B.
\item Intersection ($\cap$) - also known as the \textbf{AND} operator. An intersection is where two things meet. The intersection of the sets A and B consists of the elements that in both A and B.
\item Complement ($A^{\prime}$ or $A^{c}$) - The complement of the set A consists of all of the elements in the universal set that are not elements of A.
\end{itemize}


%--------------------------------------------------------%



\subsubsection*{Intersection}
\begin{itemize}

\item Intersection of two sets describes the elements that are members of both the specified Sets

\item The intersection is denoted $\mathcal{A\cap B}$ 
\[ \mathcal{A\cap B} = \{2\}\]

\item only one element is a member of both A and B.
\end{itemize}

%--------------------------------------------------------%


\subsection{Important Operations in Set Theory}

\begin{itemize}
\item Union ($\cup$) - also known as the OR operator. A union signifies a bringing together. The union of the sets A and B consists of the elements that are in either A or B.
\item Intersection ($\cap$) - also known as the AND operator. An intersection is where two things meet. The intersection of the sets A and B consists of the elements that in both A and B.
\item Complement ($^{c}$) - The complement of the set A consists of all of the elements in the universal set that are not elements of A.
\end{itemize}

%---------------------------------------%
\subsection*{Dice Rolls}
Consider rolls of a die. What is the universal set?

\[ \mathcal{U} = \{1,2,3,4,5,6\} \]

%--------------------------------------%
\subsection*{Worked Example}

Suppose that the Universal Set $\mathcal{U}$ is the set of integers from 1 to 9.
\[ \mathcal{U} = \{1,2,3,4,5,6,7,8,9\}, \]

and that the set $\mathcal{A}$ contains the prime numbers between 1 to 9 inclusive.

\[ \mathcal{A} = \{1,2,3,5,7\}, \]

and that the set $\mathcal{B}$ contains the even numbers between 1 to 9 inclusive.

\[ \mathcal{B} = \{2,4,6,8\}. \]


\subsubsection*{Intersection}
\begin{itemize}

\item Intersection of two sets describes the elements that are members of both the specified Sets

\item The intersection is denoted $\mathcal{A\cap B}$ 
\[ \mathcal{A\cap B} = \{2\}\]

\item only one element is a member of both A and B.
\end{itemize}
%--------------------------------------------------------%


\subsection*{symbols}
$\varnothing$,
$\forall$,
$\in$,
$\notin$,
$\cup$
%----------------------------------------------------------- %
\newpage

\section{Set Operations}
Set Operations include Set Union, Set Intersection, Set Difference, Complement of Set, and Cartesian Product.

\subsection{Set Union}
The union of sets A and B (denoted by $A \cup B$) is the set of elements which are in A, in B, or in both A and B. Hence, $A \cup B={x|x \in \mbox{ A OR x } \in B}$.

Example; $If A={10,11,12,13}$ and $B = {13,14,15}$, then $A \cup B={10,11,12,13,14,15}$. (The common element occurs only once)

%=========================================%
\subsection{Set Intersection}
The intersection of sets A and B (denoted by $AnB$) is the set of elements which are in both A and B. Hence, $A \cap B={x|x \in A AND x \in B}$.

Example - If $A={11,12,13}$ and $B={13,14,15}$, then $A \cap B={13}$.


%======================================================================================= %
\newpage
\section*{Set Operations}
\begin{itemize}
\item Union ($\cup$) - also known as the \textbf{OR} operator. A union signifies a bringing together. The union of the sets A and B consists of the elements that are in either A or B.
\item Intersection ($\cap$) - also known as the \textbf{AND} operator. An intersection is where two things meet. The intersection of the sets A and B consists of the elements that in both A and B.
\item Complement ($A^{c}$ or $A^{c}$) - The complement of the set A consists of all of the elements in the universal set that are not elements of A.
\end{itemize}

\newpage
\subsection*{Union and intersection of sets}
\begin{itemize}
\item The \textbf{union} of two sets A and B is a set containing all the elements in
either A or B (or both)
i.e.
\[A \cup B = {x / x \in A \mbox{ or } x \in B}.\]
\item The \textbf{intersection} of two sets A and B is a set containing all the elements
that are both in A and B
i.e.
\[A \cap B = {x / x \in A \mbox{ and }x \in B}\].
\item If sets A and B have no elements in common, i.e. $A \cap B = \emptyset$,then A and B
are termed \textbf{disjoint sets}.
\end{itemize}
\newpage
%--------------------------------------------------------%
\subsubsection*{Intersection}
\begin{itemize}
\item Intersection of two sets describes the elements that are members of both the specified Sets
\item The intersection is denoted $\mathcal{A\cap B}$
\[ \mathcal{A\cap B} = \{2\}\]
\item only one element is a member of both A and B.
\end{itemize}
%--------------------------------------------------------%


%======================================================================== %
\section{Set operations}
Suppose X and Y are sets. Various operations allow us to build new sets from them.
\begin{description}
\item[Union]
The union of X and Y, written $X\cup Y$, contains all the elements in X and all those in Y. Thus $A \cup B = \{1, 2, 3, red, green, blue\}$. As A is a subset of E, the set $A \cup E$ is just E.
\item[Intersection]
The intersection of X and Y, written XnY, contains all the elements that are common to both X and Y. Thus {1,2,3,red,green,blue} n {2,4,6,8,10} = {2}.
\end{description}
%---------------------------------%
\subsection{Important Operations in Set Theory}
\begin{itemize}
\item Union ($\cup$) - also known as the OR operator. A union signifies a bringing together. The union of the sets A and B consists of the elements that are in either A or B.
\item Intersection ($\cap$) - also known as the AND operator. An intersection is where two things meet. The intersection of the sets A and B consists of the elements that in both A and B.
\item Complement ($^{c}$) - The complement of the set A consists of all of the elements in the universal set that are not elements of A.
\end{itemize}
%================================= %
\subsection*{Example 1: }
\begin{verbatim}
If $U = {1, 2, 3, 4, 5, 6, 7, 8, 9, 10}$, \\
$A = {2, 4, 6, 8, 10}$, \\
$B = (1, 3, 6, 7, 8}$ \\
$C = {3, 7}$, \\
find $A \cap B$, $A \cup C$, $B \cap A^{c}$, $B \cap C^{c}$
\end{verbatim}
\begin{verbatim}
Solution:
$U = {1, 2, 3, 4, 5, 6, 7, 8, 9, 10}$\\
$A = {2, 4, 6, 8, 10}$\\
%$B = (1, 3, 6, 7, 8} $\\
$C = {3, 7}$\\
%A n B = {6, 8}
%A \cap C = {2, 3, 4, 6, 7, 8, 10}
%B n A^C = {1, 3, 7}
%B n C^C = {1, 6, 8}
\end{verbatim}
\begin{verbatim}
%> A
%[1]  1  8  9 10
%> B
%[1] 4 7 9
%> C
%[1] 1 2 3 4 9
%> D
%[1] 2 5 6
\end{verbatim}

%--------------------------------------------------------%
\subsubsection*{Intersection}
\begin{itemize}
\item Intersection of two sets describes the elements that are members of both the specified Sets
\item The intersection is denoted $\mathcal{A\cap B}$
\[ \mathcal{A\cap B} = \{2\}\]
\item only one element is a member of both A and B.
\end{itemize}
%--------------------------------------------------------%

%--------------------------------------------------------%
\subsection{Set Operations}
Consider the universal set U such that $U=\{1,2,3,4,5,6,7,8,9\}$  and the sets $A=\{2,5,7,9\}$  and  $B=\{2,4,6,8,9\}$
Perform the following binary operations
\begin{enumerate}
\item A-B
\item  A ?B
\item  $A \cap B$
\item $A\cup B$
\item $A^{c} \cup B^{c}$
\item $A^{c}\cap B^{c}$
\end{enumerate}




\end{document}
