
\documentclass[12pt]{article}
%\usepackage[final]{pdfpages}

\usepackage{graphicx}
\graphicspath{{/Users/kevinhayes/Documents/teaching/images/}}

\usepackage{tikz}
\usetikzlibrary{arrows}

\newcommand{\bbr}{\Bbb{R}}
\newcommand{\zn}{\Bbb{Z}^n}

%\usepackage{epsfig}
%\usepackage{subfigure}
\usepackage{amscd}
\usepackage{amssymb}
\usepackage{amsbsy}
\usepackage{amsthm}
\usepackage{natbib}
\usepackage{amsbsy}
\usepackage{enumerate}
\usepackage{amsmath}
\usepackage{eurosym}
%\usepackage{beamerarticle}
\usepackage{txfonts}
\usepackage{fancyvrb}
\usepackage{fancyhdr}
\usepackage{natbib}
\bibliographystyle{chicago}

\usepackage{vmargin}
% left top textwidth textheight headheight
% headsep footheight footskip
\setmargins{2.0cm}{2.5cm}{16 cm}{22cm}{0.5cm}{0cm}{1cm}{1cm}
\renewcommand{\baselinestretch}{1.3}


\pagenumbering{arabic}

\begin{document}

\subsection{Power Set}
Power set of a set S is the set of all subsets of S including the empty set. The cardinality of a power set of a set S of cardinality n is $2^n$. Power set is denoted as $P(S)$.

Example 

For a set $S={a,b,c,d}$ let us calculate the subsets 

\begin{itemize}
\item Subsets with 0 elements : $\{ \varnothing\}$ (the empty set)
\item Subsets with 1 element : {a},{b},{c},{d}
\item Subsets with 2 elements : {a,b},{a,c},{a,d},{b,c},{b,d},{c,d}
\item Subsets with 3 elements : {a,b,c},{a,b,d},{a,c,d},{b,c,d}
\item Subsets with 4 elements : {a,b,c,d}
\end{itemize}

Hence, \[P(S)=
{{\varnothing},{a},{b},{c},{d},{a,b},{a,c},{a,d},{b,c},{b,d},{c,d},{a,b,c},{a,b,d},{a,c,d},{b,c,d},{a,b,c,d}}\]
\[|P(S)|=2^4=16\]
Note: The power set of an empty set is also an empty set.

$|P({\varnothing})|=2^0=1$
\medskip

\textbf{Power set}

The power set of X, $P(X)$, is the set whose elements are all the subsets of X. Thus \[P(A) = \{ \{\}, \{1\}, \{2\}, \{3\}, \{1,2\}, \{1,3\}, \{2,3\}, \{1,2,3\}\}.\] The power set of the empty set $P(\{\})$ = $\{\{\}\}$. 

Note that in both cases the cardinality of the power set is strictly greater than that of base set: No one-to-one correspondence exists between the set and its power set. 

%Cantor proved that this in fact holds for any set (Cantor's Theorem). This is obvious for a finite set, but Cantor's ingenious proof made no reference to the set being finite; the theorem holds even for infinite sets. This was a powerful generalisation of his previous discovery that different sizes of infinity exist.




\subsubsection*{Power Sets}

\begin{itemize}

\item Consider the set A where $ A = \{w,x,y,z\}$
\item There are 4 elements in set A.

\item The power set of A contains 16 element data sets.

%Complete this
 \[  \mathcal{P}(A) = \{\{ x \}, \{ y \} \{\{ x,y \}, \{ w,y \}\}  \]

\item (i.e. 1 null set, 4 single element sets, 6 two -elements sets, 4 three element set and 1 four element set.)

\end{itemize}

%-------------------------------------------------% 

\section*{Power Sets}
\subsection*{Worked Example}
Consider the set $Z$:
\[ Z = \{ a,b,c\}  \]
\begin{itemize}
\item[Q1] How many sets are in the power set of $Z$? 
\item[Q2] Write out the power set of $Z$. 
\item[Q3] How many elements are in each element set?
\end{itemize}
%----------------------------------------------%
\subsection*{Solutions to Worked Example}

\begin{itemize}


\item[Q1] There are 3 elements in $Z$. So there is $2^3 = 8$ element sets contained in the power set.

\item[Q2] Write out the power set of $Z$.
\[ \mathcal{P}(Z) = \{ \{0\}, \{a\}, \{b\}, \{c\}, \{a,b\}, \{a,c\}, \{b,c\}, \{a,b,c\} \]

\item[Q3]
\begin{itemize}
\item[*] One element set is the null set - i.e. containing no
elements \item[$\bullet$] Three element sets have only elements \item[$\bullet$]
Three element sets have two elements \item[$\bullet$] One element set
contains all three elements \item[$\bullet$] 1+3+3+1=8
\end{itemize}
\end{itemize}
\subsection*{Exercise}
For the set $Y = \{u,v,w,x\}$ , answer the questions from the
previous exercise



\subsection{Partitioning of a Set}
Partition of a set, say S, is a collection of n disjoint subsets, say $P1,P2 \ldots Pn$  that satisfies the following three conditions :

\begin{enumerate}
\item $Pi$ does not contain the empty set.

 $[Pi \neq {\varnothing} for all 0<i\leq n]$
\item The union of the subsets must equal the entire original set.

\[P1 \cup P2 \cup \ldots \cup Pn=S\]
\item The Intersection of any two distinct sets is empty.

\[Pa \cap Pb={\varnothing}, for a\neq b where n\geq a,b\geq 0\]
\end{enumerate}

%--------------------------------------------------------%



\subsubsection*{Example}

Let $S={a,b,c,d,e,f,g,h}$
\begin{itemize}
\item One probable partitioning is ${a},{b,c,d},{e,f,g,h}$
\item Another probable partitioning is ${a,b},{c,d},{e,f,g,h}$
\end{itemize}
%-------------------------------------------------% 


\end{document}
