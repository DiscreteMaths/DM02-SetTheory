% Chapter 2
% Sets and Binary Operations
%
% Summary
% Set notation, specifying sets by the listing method and rules of inclusion method; special sets of
% numbers; empty set; cardinality; subsets, power set; set complement; binary operations on sets:
% union, intersection, difference and symmetric difference; Venn diagram, membership table; laws of
% set algebra.
%
% References: Epp Sections 5.1 (pp 231-237), 5.2, 5.3 (pp 258-264) or M&B Sections 2.1, 2.2, 2.3,2.4.
%-------------------------------------------------------------------------------------------- %
% mgmt maths


\documentclass{beamer}

\usepackage{amsmath}
\usepackage{amssymb}


\begin{document}
\begin{frame}
\frametitle{Equal and Equivalent Sets}
\Large

Difference between equal sets and equivalent sets

\begin{itemize}
\item Consider the sets \textbf{A} and \textbf{B}
\[ \boldsymbol{A} = \{ 1,2,3,4,5,6 \} \] 
\[ \boldsymbol{B} = \{1,2,3,4,5,6 \} \]
\vspace{0.2cm}
\item \textbf{A} and \textbf{B} are \textit{\textbf{equal}} sets because \textit{\textbf{all}} their
elements are precisely the \textit{\textbf{same}}.
\end{itemize}


\end{frame}
%---------------------------------------------------%

\begin{frame}
\frametitle{Equal and Equivalent Sets}
\Large
\vspace{-0.7cm}
Difference between equal sets and equivalent sets

\begin{itemize}
\item Consider the sets \textbf{C} and \textbf{D}
\[ \boldsymbol{C} = \{a,b,c,d,e,f\} \]  \[ \boldsymbol{D} = \{3,4,5,6,7,8\} \]
\vspace{0.2cm}
\item \textbf{C} and \textbf{D} are \textit{\textbf{equivalent}} sets
because the cardinality of both the sets is the same (i.e. 6.)
\item However \textbf{C} and \textbf{D} are not equal, as they are comprised of different elements.
\end{itemize}





\end{frame}
%---------------------------------------------------%

\begin{frame}
\frametitle{Equal and Equivalent Sets}
\Large


\begin{itemize}
\item Necessarily all equal sets are equivalent sets.
\item But are equivalent sets equal sets?

\item No, because equivalent sets are sets that have the \textit{\textbf{same}} cardinality but equal sets are sets that all
their elements are precisely the \textit{\textbf{same}}. 
\end{itemize}

\textbf{Example}:
\begin{itemize} \item \textbf{X}=\{p,q,r\} and \textbf{Y}=\{1,2,3\} are equivalent sets 
\item \textbf{E} =\{m,n,o,p\}
and \textbf{F}=\{m,n,o,p\} are equal and equivalent sets
\end{itemize}
\end{frame}
%---------------------------------------------------%
%---------------------------------------------------%
%SLIDE SET 2
%---------------------------------------------------%
%---------------------------------------------------%
% Opening Slide 2

\begin{frame}
\Huge
\[\mbox{Discrete Mathematics}\]
\Huge
\[\mbox{Set Theory}\]

\Large
\[\mbox{www.Stats-Lab.com}\]
\Large
\[\mbox{Twitter: @StatsLabDublin}\]

\end{frame}
%---------------------------------------------------%
%---------------------------------------------------%
% Schaum 1:4
\begin{frame}
\Large
\frametitle{Set Operations}
Let $U = \{1,2,\ldots, 9\}$ be the universal set, and let
\begin{itemize}
\item A = $\{1, 2, 3, 4, 5\}$,  
\item B = $\{4, 5, 6, 7\}$,  
\item C = $\{5, 6, 7, 8, 9\}$
\item D = $\{1, 3, 5, 7, 9\}$,
\item E = $\{2, 4, 6, 8\}$,
\item F = $\{1, 5, 9\}$.
\end{itemize}
\end{frame}
%---------------------------------------------------%
%---------------------------------------------------%
% Schaum 1:4
\begin{frame}
\frametitle{Set Operations}
\Large
\vspace{-3cm}
Find: 
\begin{itemize}
\item[(a)] A $\cup$ B and A $\cap$ B, 
\item[(b)] C $\cup$ D and C $\cap$ D, 
\item[(c)] E $\cup$ F and E $\cap$ F.
\end{itemize}

\end{frame}
%---------------------------------------------------%
%---------------------------------------------------%
% Schaum 1:4
\begin{frame}
\frametitle{Set Operations}
\Large
\vspace{-3cm}
Find: 
\begin{itemize}
\item[(a)] A $\cup$ B and A $\cap$ B, \bigskip
\item A = $\{1, 2, 3, 4, 5\}$  
\item B = $\{4, 5, 6, 7\}$  
\end{itemize}

\end{frame}
%---------------------------------------------------%
%---------------------------------------------------%
% Schaum 1:4
\begin{frame}
\frametitle{Set Operations}
\Large
\vspace{-3cm}
Find: 
\begin{itemize}
\item[(b)] C $\cup$ D and C $\cap$ D,  \bigskip
\item C = $\{5, 6, 7, 8, 9\}$
\item D = $\{1, 3, 5, 7, 9\}$
\end{itemize}

\end{frame}
%---------------------------------------------------%
%---------------------------------------------------%
% Schaum 1:4
\begin{frame}
\frametitle{Set Operations}
\vspace{-3cm}
\Large
Find: 
\begin{itemize}
\item[(c)] E $\cup$ F and E $\cap$ F. \bigskip
\item E = $\{2, 4, 6, 8\}$

\item F = $\{1, 5, 9\}$
\end{itemize}

\end{frame}
%---------------------------------------------------%
%---------------------------------------------------%
%SLIDE SET 3 : FINITE SETS
%---------------------------------------------------%
%---------------------------------------------------%
% Opening Slide 3

\begin{frame}
\Huge
\[\mbox{Discrete Mathematics}\]
\Huge
\[\mbox{Set Theory : Finite Sets}\]

\Large
\[\mbox{www.Stats-Lab.com}\]
\Large
\[\mbox{Twitter: @StatsLabDublin}\]

\end{frame}
%---------------------------------------------------%
\begin{frame}
\frametitle{Finite Sets}
\Large
\vspace{-1.2cm}
Determine which of the following sets are finite:
\begin{itemize}
\item[(a)] Set of Prime numbers %infinite.
\vspace{0.4cm}
\item[(b)] Set of two digit Prime numbers %infinite.
%\item[(a)] Lines parallel to the x axis. 
\vspace{0.4cm}
\item[(c)] Letters in the English alphabet.
\vspace{0.4cm}
\item[(d)] Integers which are multiples of 5.
\vspace{0.4cm}
%\item[(c)] Letters of the English alphabet
\item[(e)] Days of the week % finite sets.
%\item[(e)] The number of grains of rice in a ton of the grain
\end{itemize}
\end{frame}




%---------------------------------------------------%
%---------------------------------------------------%
%SLIDE SET 4
%---------------------------------------------------%
%---------------------------------------------------%
% Opening Slide 4

\begin{frame}
\Huge
\[\mbox{Discrete Mathematics}\]
\Huge
\[\mbox{Set Theory}\]

\Large
\[\mbox{www.Stats-Lab.com}\]
\Large
\[\mbox{Twitter: @StatsLabDublin}\]

\end{frame}

%---------------------------------------------------%
\begin{frame}
\frametitle{Set Theory}
\Large
\vspace{-1.8cm}
Given the set \textbf{A} is contructed as follows 
\[ [\{a, b\}, \{c\}, \{d, e, f \} ]. \]

\begin{itemize}
\item[(a)] List the elements of \textbf{A}. 
\item[(b)] Find the cardinality of \textbf{A} : $n(\boldsymbol{A})$. 
\item[(c)] Find the power set of \textbf{A}.
\end{itemize}

\end{frame}

%---------------------------------------------------%
\begin{frame}
\frametitle{Set Theory}
\Large
\vspace{-3.8cm}
\[\boldsymbol{A} = [\{a, b\}, \{c\},\{d, e, f \}]. \]
\begin{itemize}
\item[(a)] List the elements of \textbf{A}. 
\end{itemize}

\end{frame}

%---------------------------------------------------%
%---------------------------------------------------%
\begin{frame}
\frametitle{Set Theory}
\Large
\vspace{-3.8cm}
\[\boldsymbol{A} = [\{a, b\}, \{c\},\{d, e, f \}]. \]
\begin{itemize}
\item[(b)] Find the cardinality of \textbf{A} : $n(\boldsymbol{A})$. 
\end{itemize}

\end{frame}

%---------------------------------------------------%
%---------------------------------------------------%
\begin{frame}
\frametitle{Set Theory}
\Large
\vspace{-3.8cm}
\[\boldsymbol{A} = [\{a, b\}, \{c\},\{d, e, f \}]. \]
\begin{itemize}
\item[(c)] Find the power set of \textbf{A}. 
\end{itemize}

\end{frame}

%---------------------------------------------------%
%---------------------------------------------------%
%SLIDE SET 5
%---------------------------------------------------%
%---------------------------------------------------%
% Opening Slide 5

\begin{frame}
\Huge
\[\mbox{Discrete Mathematics}\]
\Huge
\[\mbox{Set Theory}\]

\Large
\[\mbox{www.Stats-Lab.com}\]
\Large
\[\mbox{Twitter: @StatsLabDublin}\]

\end{frame}
\begin{frame}
\Large
Consider the set \textbf{A}, which is a subset of the the universal set of real numbers $\mathbb{R}$
\[\boldsymbol{A} = \{4, \sqrt{2}, 2/3, -2.5, -5, 33, \sqrt{9}, \pi \}\]
Using formal set notation, write the sets of:
\begin{itemize}
\item[(a)] natural numbers in \textbf{A}
\item[(b)] integers in \textbf{A}
\item[(c)] rational numbers in \textbf{A}
\item[(d)] irrational numbers in \textbf{A}
\end{itemize}
\end{frame}
%---------------------------------------------------------- %
\begin{frame}
%0 Reveals
\Large
\vspace{-0.5cm}
\[\boldsymbol{A} = \{4,\; \sqrt{2},\; 2/3,\; -2.5,\; -5,\; 33,\; \sqrt{9},\; \pi \}\]
\vspace{-0.5cm}
\begin{itemize}
\item[(a)] natural numbers in \textbf{A}\\
\phantom{Answer: $\{4,\; 33,\; \sqrt{9}\}$}
\item[(b)] integers in \textbf{A}\\
\phantom{Answer: $\{4,\; -5, 33,\; \sqrt{9}\}$}
\item[(c)] rational numbers in \textbf{A}\\
\phantom{Answer: $\{4,\; 2/3,\; -2.5,\; -5,\; 33,\; \sqrt{9}\}$}
\item[(d)] irrational numbers in \textbf{A}\\
\phantom{Answer: $\{\sqrt{2},\; \pi\}$}
\end{itemize}
\end{frame}

%---------------------------------------------------------- %
\begin{frame}
%1 Reveals
\Large
\vspace{-0.5cm}
\[\boldsymbol{A} = \{4,\; \sqrt{2},\; 2/3,\; -2.5,\; -5,\; 33,\; \sqrt{9},\; \pi \}\]
\vspace{-0.5cm}
\begin{itemize}
\item[(a)] natural numbers in \textbf{A}\\
\hspace{1cm} \textit{Answer}: $\{4,\; 33,\; \sqrt{9}\}$
\item[(b)] integers in \textbf{A}\\
\phantom{Answer: $\{4,\; -5, 33,\; \sqrt{9}\}$}
\item[(c)] rational numbers in \textbf{A}\\
\phantom{Answer: $\{4,\; 2/3,\; -2.5,\; -5,\; 33,\; \sqrt{9}\}$}
\item[(d)] irrational numbers in \textbf{A}\\
\phantom{Answer: $\{\sqrt{2},\; \pi\}$}
\end{itemize}
\end{frame}

%---------------------------------------------------------- %
\begin{frame}
%2 Reveals
\Large
\vspace{-0.5cm}
\[\boldsymbol{A} = \{4,\; \sqrt{2},\; 2/3,\; -2.5,\; -5,\; 33,\; \sqrt{9},\; \pi \}\]
\vspace{-0.5cm}
\begin{itemize}
\item[(a)] natural numbers in \textbf{A}\\
\hspace{1cm} \textit{Answer}: $\{4,\; 33,\; \sqrt{9}\}$
\item[(b)] integers in \textbf{A}\\
\hspace{1cm} \textit{Answer}: $\{4,\; -5,\; 33,\; \sqrt{9}\}$
\item[(c)] rational numbers in \textbf{A}\\
\phantom{Answer: $\{4,\; 2/3,\; -2.5,\; -5,\; 33,\; \sqrt{9}\}$}
\item[(d)] irrational numbers in \textbf{A}\\
\phantom{Answer: $\{\sqrt{2},\; \pi\}$}
\end{itemize}
\end{frame}

%---------------------------------------------------------- %
\begin{frame}
%3 Reveals
\Large
\vspace{-0.5cm}
\[\boldsymbol{A} = \{4,\; \sqrt{2},\; 2/3,\; -2.5,\; -5,\; 33,\; \sqrt{9},\; \pi \}\]
\vspace{-0.5cm}
\begin{itemize}
\item[(a)] natural numbers in \textbf{A}\\
\hspace{1cm} \textit{Answer}: $\{4,\; 33,\; \sqrt{9}\}$
\item[(b)] integers in \textbf{A}\\
\hspace{1cm} \textit{Answer}: $\{4,\; -5,\; 33,\; \sqrt{9}\}$
\item[(c)] rational numbers in \textbf{A}\\
\hspace{1cm} \textit{Answer}: $\{4,\; 2/3,\; -2.5,\; -5,\; 33,\; \sqrt{9}\}$
\item[(d)] irrational numbers in \textbf{A}\\
\phantom{Answer: $\{\sqrt{2},\; \pi\}$}
\end{itemize}
\end{frame}
%---------------------------------------------------------- %
\begin{frame}
%4 Reveals
\Large
\vspace{-0.5cm}
\[\boldsymbol{A} = \{4,\; \sqrt{2},\; 2/3,\; -2.5,\; -5,\; 33,\; \sqrt{9},\; \pi \}\]
\vspace{-0.5cm}
\begin{itemize}
\item[(a)] natural numbers in \textbf{A}\\
\hspace{1cm} \textit{Answer}: $\{4,\; 33,\; \sqrt{9}\}$
\item[(b)] integers in \textbf{A}\\
\hspace{1cm} \textit{Answer}: $\{4,\; -5,\; 33,\; \sqrt{9}\}$
\item[(c)] rational numbers in \textbf{A}\\
\hspace{1cm} \textit{Answer}: $\{4,\; 2/3,\; -2.5,\; -5,\; 33,\; \sqrt{9}\}$
\item[(d)] irrational numbers in \textbf{A}\\
\hspace{1cm} \textit{Answer}: $\{\sqrt{2},\; \pi\}$
\end{itemize}
\end{frame}



%---------------------------------------------------%
%---------------------------------------------------%
%END OF SLIDES
%---------------------------------------------------%
%---------------------------------------------------%
\begin{frame}

End of Slide Set
\end{frame}



\section{Sets}
\begin{frame}  %GOOD
\begin{itemize}
\item A set is simply a collection of things or objects, of any kind. These objects
are called elements or members of the set. We refer to the set as an
entity in its own right and often denote it by A, B, C or D, etc.
\item If A is a set and x a member of the set, then we say $x \in A$ i.e. x ‘belongs to’
A. 
\end{itemize}
\end{frame}
%---------------------------------------------------------------------------------------------------- %
% CIS 102
\section{Introduction} %GOOD - MERGE with above
\begin{frame}
By a set we simply mean a collection or class of objects. The objects in the set are Called its
members or elements. Sets have become the basic language in which most results in mathematics
and computer science are expressed. 
In this chapter, we look at ways in which sets are specified,
how they may be represented and how they are combined to make other sets.
\end{frame}
%----------------------------------------------------------------------------------------------- %
\begin{frame}
\frametitle{Introduction to Set Theory}
\begin{itemize}
\item The symbol $\notin$ denotes the negation of ? $in$.e. $x \notin A$ means ‘x does not
belong to’ A.
\item The elements of a set, and hence the set itself, are characterised by having
one or more properties that distinguish the elements of the set from those
not in the set.
\item For example, if C is the set of non-negative real numbers, then we
might use the notation
\[C = {x | \mbox{x is a real number and }x \neq 0}\]
i.e. the set of all x such that x is a real number and non-negative.
\end{itemize}
\end{frame}

%-------------------------------------------------------------------------------------------------- %
% CHAPTER 2. SETS AND BINARY OPERATIONS 19
\begin{frame} %GOOD
\frametitle{Introduction to Set Theory}
\Large
Definition 2.1 
We use the symbol E to mean belongs to and ¢ to mean does not belong to.
Thus we write y E X to denote that (the element) y belongs to (the set) X and y §§ X to indicate
that y is not a member of the set X.
%------------ %
%IN and NOTIN symbols
%------------------------------------------------- %
\begin{frame} %GOOD
\frametitle{Introduction to Set Theory}
\Large
Examples:

\begin{itemize}
\item If A is the set of all integers then $14 \in A$ but $3.14 \notin A.$
\item Since sets are determined by their elements we say that A = B if and only
if they have the same elements.

\item $\emptyset$ represents the empty or null set, i.e. a set containing no elements.
\item The set containing everything is termed the \textbf{Universal set} and is usually
written as \textbf{U}.
\end{itemize}
\end{frame}
%EMPTY SET
%UNIVERSAL SET
\begin{frame}

\frametitle{The Empty Set}

Another set which has a special letter to denote it is the set containing no elements. This is called
the empty or null set and denoted by the symbol (D. _
Example 2.6 The set of integers m such that m2 = 5 is the empty set. We could write
{mEZ:m2=5}:@.
\end{frame}
%----------------------------------------- %

%------------------------------------------------------------------------------------------------------ %
\section{2.1 Specifying sets}
%Learning Objectives
\begin{frame}
When you have completed your study of this section, you should be able to:
\begin{itemize}
\item use set notation for specifying sets by the listing method and rules of inclusion method;
\item use and interpret the standard symbols for special sets of numbers and for the empty set.
\end{itemize}

\end{frame}
%---------------------------------------------------- %
\subsection{2.1.1 Listing method}
\begin{frame}
We usually use an upper case letter to denote a set and a lower case letter to denote a member of
the set. To specify a set, we must describe its members in an unambiguous way. One way of doing
this is to list the members of the set, separated by commas, and enclose the list in a pair of brace
brackets.
\end{frame}
%--------------------------------------- %
\begin{frame}
%Example 2.1 
\begin{itemize}
\item The set D of decimal digits can be expressed as
\[D = {0, l,2,3,4,5,6,7,8,9}\]
\item The set B of bits can be expressed as
\[B= {0,1}\]. 
\end{itemize}
\end{frame}



Example 2.2 Referring to the sets D and B defined in Example 2.1, we can say that $5 \in D$, but
5 63 B. E1
%-------------------------------------------------------------------------------------------
You will meet quite a lot of new notation in this chapter. In order for it to become familiar, it is
important to verbalise the symbol as you read it. 

So, in Example 2.2, we write $5 \in D$, but we
read this as “5 belongs to (set) D”, or “5 is in D”. Similarly, we write $5 \notin B$, but read this as
“5 does not belong to (set) B" or “5 is not in B”.
%---------------------------------------------------------------------- %
\begin{frame}
Sometimes the elements in the set are in a sequence that is easily recognised when we are given
the first few terms. In this case, instead of listing every member of the set, we can use a sequence
of three dots (called an \textbf{elipsis}) to mean “et cetera” or “and so on”, as the following example
illustrates.
\end{frame}
%------------------------------------- %
\begin{frame} %GOOD
Example 2.3 Let H be the set of integers from 1 to 100. We Write
\[H: {1,2,3,...,100}.\]
ln a similar way, we can express the set of positive odd integers as
\[K : {1,3,5,7,..  \]
\end{frame}
%------------------------------------- %
%REAL NUMBERS
%NATURAL NUMBERS
%INTEGERS

\section{Special sets of numbers}
\begin{frame}
It is convenient to denote certain key sets of numbers by a standard letter.
Definition 2.2 
The symbol Z is used to denote the set of integers," 
 denotes the set of positive integers; 
 R denotes the set of real numbers;
Q denotes the set of rational numbers {the letter Q stands for “quotient”).
\end{frame}
%-------------------------------------- %
\begin{frame}
Of these sets, we can specify the set of positive integers and the set of integers by the listing method
using elipses, as follows:
\[Z^{+} = {1,2,s,...}\]
\[2 = {o,1,-1,2,-2,3,-2,...}\]
Note that Z includes 0 and the negative whole numbers as well as the positive ones. Similarly,
\mathbb{R} contains O and the negative reals and Q contains 0 and the negative rationals, as well as the
positive ones.
\subsection{2.1.2 Rules of inclusion method}
\begin{frame}
\frametile{Rules of Inclusion}
Another way of specifying a set is by giving rules of inclusion that distinguish members of the
set from objects not in the set.

Definition 2.3 The context of the problem in which u set arises determines an underlying set,
called the uni-vet-sul set for the problem, from which the elements of the set will be drawn.
\end{frame}
%----------------------------------------- %
\begin{frame}
For example, if our subject is a set of leopards,~the universal set, explicitly stated or implied by the
context, might be all wild animals in Africa or all animals in London Zoo or all animals belonging
to the cat family.

Example 2.4 To specify the set H of Example 2.3, we could write
\[H={n \in Z : 1 \leq n \leq 100\}.\]

\end{frame}
%----------------------------------------- %
\begin{frame}
\frametitle{}
%CHAPTER 2. SETS AND BINARY OPERATIONS 20
This tells us that the universal set is \mathbb{Z} and the rule of inclusion is that n must be between 1 and
100 inclusive; the colon stands for the words such that, and so we would read this description of
H as
H is the set of integers 11 such that $1 \leq n \leq 100$.
Similarly, we could specify the set K of Example 2.3 as
\[K = {m \in Z+ : m is odd}. \]
\end{frame}
%----------------------------------------- %
\begin{frame}
Example 2.5 Let X be the set of real numbers that satisfy the equation I2 — z = O. Then we
could write
X={a:Ell¥:r2-220},
read as
X is tlle set of real numbers 1: such that 1:2 -— :1: = 04 E1



\begin{frame}
\frametitle{Even and odd integers}
Given an integer, you could probably say immediately whether it is odd or even, but how do we
define the set of even integers and the set of odd integers? To test whether an integer is even we
divide it by 2: the even integers are just those that are divisible by 2; the odd integers are just
those that are not divisible by 2. Hence the set of even integers is
{0, 2, -2,4, —4,6, —6 . . 4}
\end{frame}
%--------------------------------------------------- %
\begin{frame}
Notice that O is even, and even integers can be negative as well as positive, The set of odd integers
is
{1, -1,3, —3,5, —-5, . . 
As we saw in Chapter 1, another way of saying that an integer is divisible by 2 is to say that it is
a multiple of 2. Thus an even integer is a number that can be expressed in the form 2m, where
m Q Z. The set of even integers can therefore be expressed by the rules of inclusion method as
{Zm : m E 
This is read
“the set of all numbers of the form 2m, where m is an integer.”
\end{frame}
%--------------------------------------------------- %
\begin{frame}
\begin{itemize}
\item Any odd integer can be obtained by adding 1 to (or subtracting 1 from) some even integer. Thus
the set of odd integers can be expressed as
%{2m+l:mEZ}or{2m~l:mEZ}.
\item Other sets of integers which have as defining property that they are all multiples or powers of a
given fixed integer can be expressed in a similar way.
\end{itemize}

\end{frame}
%--------------------------------------------------- %
\begin{frame}
%Example 2.7 
\begin{itemize}
\item The set $\{.. ,-20,—10,0,10,20,...\}$ of multiples of 10 can be expressed by the rules
of inclusion method as
%\[{I00 : (1 G Z},\]
\item The set $\{1,10,100,1000,...\}$ can be expressed as ’
{10' =rez,r3 0}.n
\end{itemize}
\end{frame}
%----------------------------------------- %
\begin{frame}

%CHAPTER 2. SETS AND BINARY OPERATIONS 21
%2.2 Subsets
Learning Objectives
When you have completed your study of this section, you should be able to:
e state the condition for a set to be contained in another set as a subset and the condition for
two sets to be equal; 
\begin{itemize}
\item use subset notation correctly;
\item  say what is meant by the cardinality and the power set of a finite set;
\item  find the power set of a given finite set.
\end{itemize}
\end{frame}
%----------------------------------------- %
\begin{frame}
Introduction -
When we are considering a set, it often arises that we would like to pick out just those objects in
the set which satisfy a given condition. When we do this, we are picking a subset of the given set.
Of course, if all the objects in the original set satisfy the condition, the subset will contain the
same members as the original set. 

\end{frame}
%----------------------------------------- %
\begin{frame}
At the other extreme, it may turn out that none of the members
of the set satisfy the condition, and then our subset will be the empty set. More typically though,
some but not all of the members of the original set will satify our condition to be in the subset.
ln this section, we shall introduce a formal test for a subset and some important notation.

\end{frame}
\section{2.2.1 Notation for subsets}
%----------------------------------------- %
\begin{frame}
Definition 2.4 Given two sets A and B, the set A is said to be a subset 0fB if every element
of/1 is also an element of B. When this is the case, we write A Q B.
Notice that A Q A, for all sets A. We also regard the empty set Q) as a subset of every set.
\end{frame}
%----------------------------------------- %
\begin{frame}
Example 2.8 Let K 2 {l,3,5,7,...}. Then we can say that K Q Z+. We can also say that
Z+ Q 7/I and that Z Q R.
We can concatenate these relations between the sets, as follows:
1<gz+gzgn.n
\end{frame}
%----------------------------------------- %
\begin{frame}
The previous example illustrates a general rule that follows from the definition of subset.
Rule 2.5 IfA, B,C are sets such that A is a subset of B, and B is a subset of C, than A is also
<1 subset of C. In symbols, this becomes:
If/lQBandBQC’,thenAQC.
\end{frame}
%----------------------------------------- %
\begin{frame}
If it is true that both A Q B and B Q A, then A and B must contain the same elements. This
observation leads to the following definition of equality of two sets.
Definition 2.6 If it is true that both A Q B and B Q A, then we say that the sets A and B are
equal and write A = B.
\end{frame}
%----------------------------------------- %
\begin{frame}
Example 2.9 Suppose X = {[1, 1}, Y : {l,0} and Z = {1,l,0,1,0]. Then since each of these
sets contain just the numbers 0 and 1, we have X = Y : Z. E1
These equalities illustrate the fact that a set is determined by its elements, the method of specifi»
cation is not important; further, we may ignore repetitions of elements and also the order in which
the elements are written.
\end{frame}
%----------------------------------------- %
% CHAPTER 2. SETS AND BINARY OPERATIONS 22
\begin{frame}


Definition 2.7 IfA Q B but /l 75 B, then A we say that A is a proper subset ofB. In this case,
B contains at least one element that is not in A. This is written symbolically as A C B.
You will have noticed that the relation “Q” for sets has some of the properties of “g” for real
numbers. For example, if we have numbers 1:, y, z such that :1: § y and y f z, then we know that
z f z; also, if we have numbers a, b such that a § b and b § 11, then we know that a = b. However,
there is an important difference. 
\end{frame}
%----------------------------------------- %
\begin{frame}
For any two real numbers 2,3,1, we know that at least one of the
statements 1: 5 y and y § :0 must be true. But in the case of sets, it is easy to find examples of
pairs of subsets X,Y of the same universal set for which neither of the statements X Q Y and
Y Q X is true.
\end{frame}
%----------------------------------------- %
\begin{frame}
Example 2.10 The sets X 1: {1,3,5} and Y = {1,2,6} are both subsets of the set of decimal
digits D. Clearly, neither the statement X Q Y nor the statement Y Q X is true. U
Definition 2.8 Let A,B be sets such that neither of the statements A Q B nor the statement
B Q A is true. Then we say that the sets A, B are noncomparable.
\end{frame}
%----------------------------------------- %
% 1.3 Sub-sets
\begin{frame}

If A and B are two sets and all the elements of A also belong to B then it
can be said that:
A is contained in B
or A is a sub-set of B
or B contains A.
These expressions are all equivalent and may be symbolically written as
A ? B.
\end{frame}
%----------------------------------------- %
\section{2.2.2 Cardinality of a set}
Definition 2.9 A set is called finite when it contains a finite numher of elements and otherwise
it is called infinite. The number of distinct elements in u finite set S is called its cardinality.
Notice that we use the word distinct in mathematics to mean “distinguishable” or“difierent".

\end{frame}
%----------------------------------------- %
Example 2.11 The sets X,Y, Z of Example 2.9 each have cardinality 2, the set H = {n € ZZ : l §
n S 100} has cardinality 100, whereas Z, Q, and TR are all examples of infinite sets. El
Note that since the empty set contains no elements, it has cardinality 0.
\section{2.2.3 Power set}
Counting subsets
Now suppose we are given a finite set S. We might want to ask how many distinct subsets does S
contain? ln order to count all possible subsets of a set of given cardinality, we shall show how We
can code each of them with a binary string. Consider the following problem.

%----------------------------------------------------------------------------------- %
\section{EXAMPLE}
Example 2.12 The College Refectory is offering three vegetables, beans, carrots and tomatoes to
accompany your main dish. You may, of course, not wish to order any of these, but if you do, you
may choose any onc, two or three of them. How many different possibilities arise‘?
Suppose we code the possible choices for your order using a. 3~bit binary string. We record 1 if you
choose an item and a 0 if you reject it. The first bit represents your decision on beans, the second
on carrot-s and the third on tomatoes. But each possible order can also be regarded as a subset
of the set V : {b,c,t} (where b = beans, c = carrots and t : tomatoes). The correspondence
between the subsets and the 3-bit codes is given in the table below.
subset] 0 {b} {C} {r} {b,¢} {b,t} {qt} {b,¢,t}
code |000 100 010 001 110 101 011 111

%---------------------------------------------------------------------------------------- %
Thus there are in all 8 choices, each represented by a different 3-bit binary string. Further, each
possible 3~bit binary string corresponds to a different subset, so that the table above shows a
one-to-one correspondence between the subsets of V and the set of all 3-bit binary strings. D
We can extend this method to code the subsets of any finite set, by using binary strings of an
appropriate length.
\end{frame}
%--------------------------------------------------- %
\begin{frame}


Example 2.13 In order to code the subsets of the set D = 6,7, 8, 9}, we must use
the set of 10-bit binary strings. T hen, for example, the subse of D is represented by
5“:-“H
,»_..¢=
F’.-»
1-‘Io
5*-”<;.=
§”'>z=-
:59‘

%------------------------------------------------ %
% CHAPTER 2. SETS AND BINARY OPERATIONS 23
the string 1111100000; and the subset of D represented by the string 1100010101 is {0, 1, 5,7, 9}.
Cl
Now we can generalise this idea, to give a formula for the number of subsets of a set containing
n elements, where n is any positive integer. Every subset of a. set containing n elements can
be represented in a unique way as an n-bit binary string. Conversely, every n-bit binary string
represents a unique subset. So there are the same number of subsets as there are n~bit binary
strings. We shall see later in the course that this number is 2". This implies the following result:
%---------------------------------------------- %
Theorem 2.10 A set with n elements has exactly 2" subsets.
Definition 2.11 We call the collection or family of all subsets ofa set S the power set of S and
denote it by 73(5).
%--------------------------------------------- %
The term power set derives from the fact that when the cardinality of S is n, then the cardinality
of ’P(.5') is 2“. Notice that both 0 and S are members of'P(S); they correspond respectively to the
binary strings in which each bit is 0 and in which each bit is 1.
The situation in which we have a set whose elements are themselves sets calls for some care in the
use of notation.
%---------------------------------------------------- %
Example 2.14 Let D : {0,1,2,...,9}. Then we have seen that we write the statement “5 is an
element of S” as
5 E 5.
The set {5} denotes the subset of D containing just the element 5‘ We express the Tact that {5}
is a subset of D by writing
{5} Q D
However, {5} is a member of the family of subsets of D, that is, the powerset of D. We express
this fact by writing
{5} € 77(1)).
Notice also, that the subset {0} is not the empty set: it is the subset of D containingjust the digit
[]_ El


\section{1.6 Differences and complements}
If A and B are sets then the difference set A - B is the set of all elements
of A which do not belong to B.
If B is a sub-set of A, then A - B is sometimes called the complement of
B in A. When A is the universal set one may simply refer to the
complement of B to denote all things not in B. The complement of a set A
is denoted as Ac or A-.
Note: De Morgan’s Theorems:
(A ? B)c = Ac ? Bc
(A ? B)c = Ac ? Bc
The above relationships are most easily confirmed by using a Venn
diagram (see below) to indicate that both sides of the above equations
amount to the same areas of the diagram.
76 Management mathematics
%------------------------------------------------------------------------- %

\section{1.7 Venn diagrams}
Often the relationships that exist between sets can best be shown using a
Venn diagram. To construct a Venn diagram we let a certain region, usually
a rectangle, represent the universal set. 

This rectangle is often implied
by the constraints of the page and only in those circumstances where its
boundary is important is the rectangle drawn (see the diagrams below for
example). Individual sets are then represented by regions, often circles,
within this rectangle. One can then easily depict intersections, unions,
complements, etc. on the diagram. For example:

%------------------------------------------------------------------------- %
\section{2.3 Operations on Sets}
\frametitle{Learning Objectives}
When you have completed your study of this section, you should be able to:
\begin{itemize}
\item define the set complement of a set and find the complement of a given set;
\item define the binary operations of union, intersection, set difference and symmetric difference of
two sets and illustrate each of these operations by a Venn diagram;
\item  find the union, intersection, set difference and symmetric difference of two given sets;
\item  state and use the commutative, associative and distributive laws for set union and intersec-
tion;
\item  use membership tables and Venn diagrams to establish relations between given sets.
\end{itemize}

\end{frame}
%------------------------------------------------------------------------ %
Introduction
In this section, we suppose that A, B,C are subsets of some universal set Ll. We shall define
operations that can be performed on these subsets to yield other subsets of M. We shall make
use of a. pictorial representation of sets, called Venn diagrams and also an equivalent, but more
general, method of defining subsets of a universal set by using membership tables. We consider
how both these methods of representation can be used to verify relations between sets.



CHAPTER 2. SETS AND BINARY OPERATIONS 24
2.3.1 The complement of a set
Definition 2.12 The set of elements 0fLJ that are not in A is called the complement of A,
denoted by A’. Thus
A’ r: {:2 1 :1: ¢ A}.

%--------------------------------------------------------------- %
Common alternative notations found in textbooks for the complement of A are ~ A and Z.
We illustrate the relation between the sets A and A’ in a Venn diagram in Figure 2.1. In drawing a
Venn diagram, we assume that all members of the universal set are contained within the rectangular
frame of the picture. We subdivide the rectangle to depict subsets of the universal set. There are
two conventions:
1. no element is depicted as lying on any boundary line of a set;
2. some regions of the diagram may contain no elements.
We usually shade the region of the diagram containing all elements in the indicated set. 
%---------------------------------------------------------------- %
i
lg.
Figure 2.1.
Note the following two special cases of complements.
Ll'=lland W=Z»l.
Notice also that for any subset A Q Ll, we have
(/ll)’ = {:0 1 1- Q A'} == A.

%----------------------------------------------------------------- %
\subsection{1.4 The order of sets: finite and infinite sets}
A set is said to be finite if it contains only a finite number of elements;
otherwise the set is an infinite set. The number of elements in a set A is
called the order of A and is denoted by ¦A¦ or n(A) or nA.
Examples:
The set of all integers is an infinite set.
The set of days in a week has order 7.

%------------------------------------------------------------------ %
\section{1.5 Union and intersection of sets}
The union of two sets A and B is a set containing all the elements in
either A or B (or both)
i.e. A ? B = {x / x ? A or x ? B}.
The intersection of two sets A and B is a set containing all the elements
that are both in A and B
i.e. A ? B = {x / x ? A and x ? B}.
If sets A and B have no elements in common, i.e. A ? B = f,then A and B
are termed disjoint sets.
The above notation can be extended into the case of a family of sets (e.g.
Ai, i = 1,2,.....k). Thus the union of the family is
? Ai = {x / x ? Ai for some i = 1,2,...,k}
i = 1, 2, ..., k
The intersection of the family is:
? Ai = {x / x ? Ai for every i = 1,2, ..., k}
i =1, 2, ..., k
Example:
If A = {1,3,5,7} and B = {1,2,3,4,5} then A ? B = {1,2,3,4,5,7} and
A ? B = {1,3,5}
%-------------------------------------------------------- %
\subsection{Activity 1.1}
\begin{frame}If A = {a,b,c,d,e,f}, B = {a,e,g,h,j}, C = {b,c,f,g} what are the following sub-sets?
\begin{itemize}
\item[(i.)] A ? B
\item[(ii)] B ? C
\item[(iii)] A ? B^c
\item[(iv.)] A ? (B ? C).
\end{itemize}
\end{frame}
%----------------------------------------------------------- %
\section{2.3.2 Binary operations on sets}
The remaining operations that we define in this section combine two subsets and are in consequence
known as binary operations.
Definition 2.13 The set of elements that are in A or in B (including the elements that are in
both sets) is called the union of A and B, and denoted by A U B. Thus we have
/lUB={ZZ12EA07‘IEB(OTl7Oll'l)}.

Definition 2.14 The set of elements that are in both A and B is called the intersection of A
and B, and denoted by AU B. Thus we have
AfiB={z::|:EAana':c€B}.
§
I  I J u
A U B is shaded region A F1 B is shaded region
Figure 2.2.


%------------------------------------------------------------------- %
% CHAPTER 2 SETS AND BINARY OPERATIONS 25
Definition 2.15 The set of elements that are in A but not in B is called the set difierence of
A and B, and denoted by A —- B. Thus we have
A—B:{a;:zEAand2:¢B}.
Definition 2.16 The set of elements that are in A or in B, but not in both, is called the sym-
metric difierence of A and B, and denoted by A63 B. Thus we have
A®B:{;z'::cEAor::EB,butnotboth}.
There are several ways of expressing the symmetric difference in terms of the other binary opera-
tions we have defined. For example
A93 = (A—B)U(B—/1)
= (.AUB)~(BF1A).
l 7 U l   M
A — B is shaded region A ® B is shaded region
Figure 23.
Notice that in drawing a Venn diagram to illustrate two sets in general, we depict the sets as
overlapping, as shown in Figures 2.2 and 2.3. Drawn in this way, the boundaries of the sets
partition the Whole area (representing Z1) into four discrete regions. We are not saying that in
every example there are necesarily elements in each of these regions; in any particular example
it may happen that one or more of the regions is empty. In particular, if the region representing
A H B is empty, then A and B are said to be disjoint subsets.
%------------------------------------- %
Example 2.15 Suppose Ll 2 Z, A : {1,3,5,7, 9}, B 2 {2,4,5,6,7,8} and C : {Q}4}~ Then
Au B = {1,2,3,4,5,s,7,s,9}; Ann = {an}; A—B = {1,3,9}; B—A = {2,4,e,s} and
A  E = {'l,3,9,2,4,6,8}. However, A O C = U, so we can say that the subsets A and C are
disjoint. D

%-------------------------------------- %
It follows directly from the definitions of union and intersection that A O B is a subset of each
of the sets A and B; similarly, both A and B are subsets of A U B. You can easily verify these
statements from the Venn diagrams shown above.

\subsection{2.3.3 Laws for binary operations}
Rule 2.17 (Identity and complement laws). Let A be a subset ofa universal set Z1. Then
(i) Afildflfl andAU@=A;
(ii) AOL! :A andAUZ1 :11;
(m) AH/5l'=Ql andAUA’=Ll. D
Examples of binary operations on the real numbers are addition, multiplication and subtraction
(division is a binary operation on the non-zero reals). Addition and multiplication obey certain
rules that are so familiar that we tend to take them for granted. For example, we know that for
any two real numbers a:,y,
:1:+y=y+:c and zyzyx.


%------------------------------------------------------------------------------- %
% PAGE 26
% CHAPTER. 2. SETS AND BINARY OPERATIONS 26
We say that the operations of addition and multiplication are commutative, because we can
commute (or interchange) the relative positions of :2 and y without changing the value of the sum
or product. On the other hand, subtraction is not a commutative operation because the statement
"1 -— y = y ~ r” is not true for all values of z and y. You will encounter another example of a
non~commutative operation when we come to study matrix multiplication later in this course.
In the case of the set operations union and intersection, however, it is clear from their definitions,
and also from the Venn diagrams above, that A U B = B UA and A H B : B F1 A, for all sets A, B_.
So we have:
Rule 2.18 (Commutative laws). For all subsets A and B of a universal set L1, we have
(i) AF1B= BOA; (ii) AUB:BU/l. D
%-------------------------------------------------------------------------------- %
\section{2.3.4 Membership tables}
Before considering binary operations on more than two sets, it is useful to develop an alternative
method to Venn diagrams for illustrating and verifying our results.
We have already noted that in drawing a Venn diagram to illustrate two subsets A, B of a universal
set Z/I, the boundaries of A and B partition the whole area (representing U) into four regions,
labelled Ra, Rb, Ra and Rd respectively in Figure 2.4 below.
U
Rn
Rb
l B
Figure 2.4.

%----------------------------------- %

We new give each of these regions a unique 2-bit binary code, using the following rule:
e the first bit is l if the region is inside the set A, and is 0 otherwise;
0 the second bit is l if the region is inside the set B, and is O otherwise.
For example, the region Ra is not in A and not in B, so we give it the binary code O0; R1, is not
in A, but it is in B, so we give region R1, the code U1, and so on. We record the codes for these
four regions in the following table [Figure 2.5). It is easy to interpret a given 2~bit binary code as
a region; for example, 10 codes the region that is in A but not in B.
region
tututobo
as vb
»-¢>-¢<:ob>
»-\o»-‘ow
Figure 2.5.
Notice that in the column indexed by set A in Figure 2.5, the entry 1 appears just in the rows for
regions RC and Rd, the two regions that comprise set A. Similarly, the entry 1 in the B column
occurs just in the rows corresponding to regions R1, and Rd, the regions comprising set B. We call
the column corresponding to set A the membership table for A.
We can construct membership tables for each of the subsets formed by combining the sets A and B
by one of the binary operations U, O, — or Q9 in a. similar way. To determine the membership table
for A ('\ B, note that this subset corresponds just to the region coded 11 [see Figure 2.2). Thus
the column for A F1 B has a 1 in the row coded 11 and a zero in each of the other rows. The set

%-------------------------------------------------------------------------------------------- %
%PAGE 27
AU B contains all elements in A or in B or in both. Hence we enter a 1 in the rows corresponding
to each of the regions coded 10, 01 and 11, and enter O in the row coded 00 (see Figure 2.2). The
resiilting membership tables are shown in Figure 2.6.
Figure 2.6.
%---------------------------------------------------------------------------------------------- %
We can use either Venn diagrams or membership tables to prove the following two laws relating
to set complements. They are due to the British mathematician Augustus De Morgan, who was a
Professor of Mathematics at London University in the later part of the nineteenth century.

>-1»-00>
+-»c>>--cm
13>
i-ooc:D
KI1
lb
>-1»->-c>C
W
%------------------------------------------------------------------------------------ %
\section{De Morgan‘s laws}
Rule 2.19 (De M0rgan’s laws). Lei A,B be subsets ofa universal mu, than
(il (.4 fl B)’ : A’ U B’;
(ii) (Al U B)’ I /l’ H B’.
These laws are very important and students often get them wrong, so let us also express them in
words. The first law says that the complement of the intersection of two sets is the union of their
complements; the second law says that the complement of the union of two sets is the intersection
of their complernerits.
As an illustration of how we use membership tables to prove a relation between two sets, we use
this method to prove (i).

%--------------------------------------------------------------------------------- %
\begin{frame}
\frametitle{Example 2.16}
We need to construct and compare columns for $(A \cap B)^{c}$ and $A’ \cup B’$. To find
the column for (A O B)’, we first obtain the column for A O B (see Figure 2.6) and then take
its complement. Recollect that the complement of a given set comprises just those regions of the
universal set that are not in the given set. Thus to obtain the membership table for (A I": B)’, we
center l in each row in which AGE has O, and U in each row in which the AH B has a 1. This gives
the following table.
\end{frame}
%-------------------------------------------------------------------------------------- %
To construct a column for A’ U B’, we first construct columns for A’ and B’ by taking the comple~
moms of the rows for A and B respectively. Now the union of two sets contains any region that is
in at least one of these sets. Thus we construct the column for A’ U B’ by putting a l in any row
in which either A’ or B’ has a 1, and O just in the row[s) where both A’ and B’ have a O (the “set
union rule” in Figure 2.6). This gives the following table.
»-»--\oc>fi>. i_1>-\<:>¢j1>
>-Q»-4013; ._=@,_om
cc»-i-\ is
»-coo)
c>»—o>- ‘I’
f\
I>>
(:>>—4D-4!-—\ ‘3’_"_““g
‘<
Since the columns corresponding to (AF: B)’ and /l’UB’ are identical, the Venn diagram for each of
these subsets will comprise precisely the same regions ofbl. Hence if at E [AfiB)’, then :1: 6 A’ UB’
and converselyi Thus these sets are equal, proving De Morgans law  U



