%\documentclass[11pt, a4paper,dalthesis]{report}    % final
%\documentclass[11pt,a4paper,dalthesis]{report}
%\documentclass[11pt,a4paper,dalthesis]{book}

\documentclass[11pt,a4paper,titlepage,oneside,openany]{article}

\pagestyle{plain}
%\renewcommand{\baselinestretch}{1.7}

\usepackage{setspace}
%\singlespacing
\onehalfspacing
%\doublespacing
%\setstretch{1.1}

\usepackage{amsmath}
\usepackage{amssymb}
\usepackage{amsthm}
\usepackage{multicol}

\usepackage[margin=3cm]{geometry}
\usepackage{graphicx,psfrag}%\usepackage{hyperref}
\usepackage[small]{caption}
\usepackage{subfig}

\usepackage{algorithm}
\usepackage{algorithmic}
\newcommand{\theHalgorithm}{\arabic{algorithm}}

\usepackage{varioref} %NB: FIGURE LABELS MUST ALWAYS COME DIRECTLY AFTER CAPTION!!!
%\newcommand{\vref}{\ref}

\usepackage{index}
\makeindex
\newindex{sym}{adx}{and}{Symbol Index}
%\newcommand{\symindex}{\index[sym]}
%\newcommand{\symindex}[1]{\index[sym]{#1}\hfill}
\newcommand{\symindex}[1]{\index[sym]{#1}}

%\usepackage[breaklinks,dvips]{hyperref}%Always put after varioref, or you'll get nested section headings
%Make sure this is after index package too!
%\hypersetup{colorlinks=false,breaklinks=true}
%\hypersetup{colorlinks=false,breaklinks=true,pdfborder={0 0 0.15}}


%\usepackage{breakurl}

\graphicspath{{./images/}}

\usepackage[subfigure]{tocloft}%For table of contents
\setlength{\cftfignumwidth}{3em}

\input{longdiv}
\usepackage{wrapfig}


%\usepackage{index}
%\makeindex
%\usepackage{makeidx}

%\usepackage{lscape}
\usepackage{pdflscape}
\usepackage{multicol}

\usepackage[utf8]{inputenc}

%\usepackage{fullpage}

%Compulsory packages for the PhD in UL:
%\usepackage{UL Thesis}
\usepackage{natbib}

%\numberwithin{equation}{section}
\numberwithin{equation}{section}
\numberwithin{algorithm}{section}
\numberwithin{figure}{section}
\numberwithin{table}{section}
%\newcommand{\vec}[1]{\ensuremath{\math{#1}}}

%\linespread{1.6} %for double line spacing

\usepackage{afterpage}%fingers crossed

\newtheorem{thm}{Theorem}[section]
\newtheorem{defin}{Definition}[section]
\newtheorem{cor}[thm]{Corollary}
\newtheorem{lem}[thm]{Lemma}

%\newcommand{\dbar}{{\mkern+3mu\mathchar'26\mkern-12mu d}}
\newcommand{\dbar}{{\mkern+3mu\mathchar'26\mkern-12mud}}

\newcommand{\bbSigma}{{\mkern+8mu\mathsf{\Sigma}\mkern-9mu{\Sigma}}}
\newcommand{\thrfor}{{\Rightarrow}}

\newcommand{\mb}{\mathbb}
\newcommand{\bx}{\vec{x}}
\newcommand{\bxi}{\boldsymbol{\xi}}
\newcommand{\bdeta}{\boldsymbol{\eta}}
\newcommand{\bldeta}{\boldsymbol{\eta}}
\newcommand{\bgamma}{\boldsymbol{\gamma}}
\newcommand{\bTheta}{\boldsymbol{\Theta}}
\newcommand{\balpha}{\boldsymbol{\alpha}}
\newcommand{\bmu}{\boldsymbol{\mu}}
\newcommand{\bnu}{\boldsymbol{\nu}}
\newcommand{\bsigma}{\boldsymbol{\sigma}}
\newcommand{\bdiff}{\boldsymbol{\partial}}

\newcommand{\tomega}{\widetilde{\omega}}
\newcommand{\tbdeta}{\widetilde{\bdeta}}
\newcommand{\tbxi}{\widetilde{\bxi}}



\newcommand{\wv}{\vec{w}}

\newcommand{\ie}{i.e. }
\newcommand{\eg}{e.g. }
\newcommand{\etc}{etc}

\newcommand{\viceversa}{vice versa}
\newcommand{\FT}{\mathcal{F}}
\newcommand{\IFT}{\mathcal{F}^{-1}}
%\renewcommand{\vec}[1]{\boldsymbol{#1}}
\renewcommand{\vec}[1]{\mathbf{#1}}
\newcommand{\anged}[1]{\langle #1 \rangle}
\newcommand{\grv}[1]{\grave{#1}}
\newcommand{\asinh}{\sinh^{-1}}

\newcommand{\sgn}{\text{sgn}}
\newcommand{\morm}[1]{|\det #1 |}

\newcommand{\galpha}{\grv{\alpha}}
\newcommand{\gbeta}{\grv{\beta}}
%\newcommand{\rnlessO}{\mb{R}^n \setminus \vec{0}}
\usepackage{listings}

\interfootnotelinepenalty=10000

\newcommand{\sectionline}{%
  \nointerlineskip %%- \vspace{\baselineskip}%
  \hspace{\fill}\rule{0.5\linewidth}{.7pt}\hspace{\fill}%
  \par\nointerlineskip %%- \vspace{\baselineskip}
}

\renewcommand{\labelenumii}{\roman{enumii})}

\begin{document}









\end{document}


NOTATIONS FOR A SET:

A set can be represented by two methods:
1.	ROSTER METHOD
2.	BUILDER METHOD
ROSTER METHOD:

In this method the elements of a set are separated by commas and are enclosed within curly brackets { }. For example:

A = {1, 2, 3, 4, 5, 6} is a set of numbers.

B = {Sunny, Joy, Kartik, Harish, Girish} is a set of names.

C = {a, e, i, o, u} is a set of vowels.

D = {apple, banana, guava, orange, pear} is a set of fruits.

Listing the elements in this way is called Roster method. In this method, it is not necessary for the elements to be listed in a particular order. The elements of the set can be written just plainly, separated by commas and in any order.


BUILDER METHOD:

This method is also called Property method. In Builder method, a set is represented by stating all the properties which are satisfied by the elements of that particular set only.

For example:

If A is a set of elements less than 0, then in Builder method it will be written as

A = {x: x < 0}, this statement is read as "the set of all x such that x is less than 0"

If A is a set of all real numbers less than 7, then in Builder method it is written as A = {x   R: x < 7}

Similarly,

A = {2i: i is an integer} is a set of all even integers.

A = {x   R: x ? 2} is a set of all real numbers except 2.

A = {x   R: x > 3 and x < 7} is a set of real numbers greater than 3 but less than 7.

A = {x   Z: x > 6} is a set of integers greater than 6.

A = {x   Z: 2x + 1} is a set of all odd integers.




% \item The Fibonacci sequence $f_n$ is defined recursively by the rule
%   \begin{equation*}
%     \begin{cases}
%       f_0&=0\\
%       f_1&=1\\
%       f_n&=a_{n-1}+a_{n-2}
%     \end{cases}
%   \end{equation*}

%   \begin{enumerate}
%   \item
%     Write a program to evaluate the Fibonacci sequence and hence evaluate $f_{50}$.
%   \item
%     Let the sequence $g_n$ be defined as the ratio
%     \begin{equation*}
%       g_n = \frac{f_{n+1}}{f_n}
%     \end{equation*}
%     Write a program to evaluate the first $50$ terms of the sequence $g_n$.
%   \item
%     Assumming that the sequence $g_n$ has a limit $\phi$, find this limit.
%   \end{enumerate}



% \pagebreak
%   \begin{center}
%     \textbf{Assignment 1}
%     Due Tuesday Week 4 (Sept $28^th$)
%   \end{center}

%   Write an Octave function which computes square roots using Heron's method to as high an accuraccy as possible. The function should be called ``sqr\_root'' and must take the following form
% \lstset{language=Octave}
% \begin{lstlisting}[title=sqr\_root.m,frame=single]
% function X=sqr_root(D)
%     ....
% endfunction
% \end{lstlisting}


%--------------------------------- %
\section*{The Universal Set and the Empty Set}
\begin{itemize}
	\item The first is the \textbf{\textit{universal set}}, typically denoted $U$. This set is all of the elements that we may choose from. This set may be different from one setting to the next. 
	
	\item For example one universal set may be the set of all real numbers, denoted $\mathbb{R}$, whereas for another problem the universal set may be the whole numbers $\{0, 1, 2,\ldots\}$.
	
	\item The other set that requires consideration is called the \textit{\textbf{empty set}}. The empty set is the unique set is the set with no elements. We write this as $\{ \}$ and denote this set by $\emptyset$.
\end{itemize}

%========================================================================================= %
\newpage
\section*{Rules of Inclusion, Listing and Cardinality}
For each of the following sets, a set is specified by the rules of inclusion method and listing method respectively. Also stated is the cardinality of that data set.
\subsection*{Worked example 1}
\begin{itemize}
\item $\{ x : x $ is an odd integer $ 5 \leq x \leq 17 \}$
\item $x = \{5,7,9,11,13,15,17\}$
\item The cardinality of set $x$ is 7.
\end{itemize}

\subsection*{Worked example 2}
\begin{itemize}
\item $\{ y : y $ is an even integer $ 6 \leq y < 18 \}$
\item $y = \{6,8,10,12,14,16\}$
\item The cardinality of set $y$ is 6.
\end{itemize}

\subsection*{Worked example 3}
A perfect square is a number that has a integer value as a
square root. 4 and 9 are perfect squares ($\sqrt{4} = 2$,
$\sqrt{9} = 3$).
\begin{itemize}
\item $\{ z : z $ is an perfect square $ 1 < z < 100 \}$
\item $z = \{4,9,16,25,36,49,64,81\}$
\item The cardinality of set $z$ is 8.
\end{itemize}

\newpage

\subsection*{Exercises}
For each of the following sets, write out the set using the listing method.
Also write down the cardinality of each set.

\begin{itemize} 
\item $\{ s : s $ is an negative integer $ -10 \leq s \leq 0 \}$
\item $\{ t : t $ is an even number $ 1 \leq t \leq 20 \}$
\item $\{ u : u $ is a prime number $ 1 \leq u \leq 20 \}$
\item $\{ v : v $ is a multiple of 3 $ 1 \leq v \leq 20 \}$
\end{itemize}
%-------------------------------------------------% 
\newpage
\section*{Power Sets}
\subsection*{Worked Example}
Consider the set $Z$:
\[ Z = \{ a,b,c\}  \]
\begin{itemize}
\item[(i)] How many sets are in the power set of $Z$? 
\item[(ii)] Write out the power set of $Z$. 
\item[(iii)] How many elements are in each element set?
\end{itemize}
%----------------------------------------------%
\subsection*{Solutions to Worked Example}

\begin{itemize}


\item[(i)] There are 3 elements in $Z$. So there is $2^3 = 8$ element sets contained in the power set.

\item[(ii)] Write out the power set of $Z$.
\[ \mathcal{P}(Z) = \{ \emptyset, \{a\}, \{b\}, \{c\}, \{a,b\}, \{a,c\}, \{b,c\}, \{a,b,c\} \} \]

\item[(iii)]
\begin{itemize}
\item[$\bullet$] One element set is the null set - i.e. containing no
elements \item[$\bullet$] Three element sets have only elements \item[$\bullet$]
Three element sets have two elements \item[$\bullet$] One element set
contains all three elements \item[$\bullet$] 1+3+3+1=8
\end{itemize}
\end{itemize}
\subsection*{Exercise}
For the set $Y = \{u,v,w,x\}$ , answer the questions from the
previous exercise


%------------------------------------------------------%

\section*{Complement of a Set}
%(2.3.1) 
Consider the universal set $U$ such that
\[U=\{2,4,6,8,10,12,15\} \]
For each of the sets $A$,$B$,$C$ and $D$, specify the complement sets.
{
	
\begin{center}
\begin{tabular}{|c|c|}
  \hline
Set &\phantom{sp} Complement \phantom{sp}\\
\hline \phantom{sp} $A=\{4,6,12,15\}$ \phantom{sp} &
$A^{\prime}=\{2,8,10\}$ \\ \hline $B=\{4,8,10,15\}$ & \\ \hline
$C=\{2,6,12,15\}$ & \\ \hline $D=\{8,10,15\}$ & \\ \hline

\end{tabular}
\end{center}
}
%-------------------------------------%
 % Binary Operations on Sets (2.3.2)
 % Union , Intersection, Symmetric Difference
 % Set Difference

%======================================================================================= %
\newpage
\section*{Set Operations}
\begin{itemize}
	\item Union ($\cup$) - also known as the \textbf{OR} operator. A union signifies a bringing together. The union of the sets A and B consists of the elements that are in either A or B.
	\item Intersection ($\cap$) - also known as the \textbf{AND} operator. An intersection is where two things meet. The intersection of the sets A and B consists of the elements that in both A and B.
	\item Complement ($A^{\prime}$ or $A^{c}$) - The complement of the set A consists of all of the elements in the universal set that are not elements of A.
\end{itemize}

\subsection*{Exercise}
Consider the universal set $U$ such that
\[U=\{1,2,3,4,5,6,7,8,9\} \] 
and the sets
\[A=\{2,5,7,9\} \] 
\[B=\{2,4,6,8,9\} \]

\begin{multicols}{2}
\begin{itemize}
	\item[(a)] $A-B$
	\item[(b)] $A \otimes B$
	\item[(c)] $A \cap B$
	\item[(d)] $A \cup B$
	\item[(e)] $A^{\prime} \cap B^{\prime}$
	\item[(f)] $A^{\prime} \cup B^{\prime}$
\end{itemize}
\end{multicols}

%======================================================================================= %
\newpage

\section*{Venn Diagrams}

Draw a Venn Diagram to represent the universal set
$\mathcal{U} = \{0,1,2,3,4,5,6\}$ with subsets:\\
$A = \{2,4,5\}$\\
$B = \{1,4,5,6\}$\\

\noindent Find each of the following
\begin{itemize}
\item[(a)] $A \cup B $
\item[(b)] $A \cap B $
\item[(c)] $A-B$
\item[(d)] $B-A$
\item[(e)] $A \otimes B$
\end{itemize}
\newpage



%------------------------------------------------------------------------%
\begin{itemize}
	\item[(i)] Describe the following set by the listing method
	\[ \{ 2r+1 : r \in Z^{+} and r \leq 5  \} \]
	\item[(ii)] Let A,B be subsets of the universal set U.
	
	
\end{itemize}

%===================================================================================== %
\subsection*{Question 1}

\begin{itemize}
	\item $\{ s :  \mbox{ s is an odd integer and } 2 \leq s \leq 10 \}$
	\item $\{ 2m :  m \in Z \mbox{ and }5 \leq m \leq 10 \}$
	\item $\{ 2^t :  t \in Z \mbox{ and } 0 \leq t \leq 5 \}$
\end{itemize}

%=======================+============================================================== %
\subsection*{Question 2}

\begin{itemize}
	\item $\{12,13,14,15,16,17\}$
	\item $\{0,5,-5,10,-10,15,-15,.....\}$
	\item $\{6,8,10,12,14,16,18\}$
\end{itemize}

\subsection*{Question 7 : Membership Tables}
Using membership tables
\begin{tabular}{|ccc|c|c|c|}
	\hline
	% after \\: \hline or \cline{col1-col2} \cline{col3-col4} ...
	A & B & C & x & y & z \\\hline
	0 & 0 & 0 & 1 & 1 & 1 \\
	0 & 0 & 1 & 0 & 0 & 1 \\
	0 & 1 & 0 & 0 & 0 & 1 \\
	0 & 1 & 1 & 0 & 0 & 1 \\
	1 & 0 & 0 & 1 & 0 & 1 \\
	1 & 0 & 1 & 1 & 0 & 1 \\
	1 & 1 & 0 & 0 & 0 & 1 \\
	1 & 1 & 1 & 1 & 0 & 1 \\
	\hline
\end{tabular}
\begin{itemize}
	\item[(i)] Draw a venn diagram to show three subsets A,B and C of a universal set U intersecting in
	the most general way?
	\item[(ii)] How are sets $X$ and $Z$ related?
	\item[(iii)] Can you describe each of the subsets X,Y and Z in terms  of the
	sets A,B,C using the operations union intersection and set complement.
\end{itemize}
%================================================================ %
\subsection*{Question 8}
\begin{itemize}
	
	\item[(i)] 
	
	\item[(ii)]
	
	\item[(iii)]
	
\end{itemize}

\end{document}



   




%-------------------------------------% Ellipsis

When using Ellipsis, it should be clear what the pattern is

%-------------------------------------%


%-----------Reference to section 2.2.3 Power Sets




%-------------------------------------%

 %----(Reference to Section 2.2.2 Cardinality)



%-------------------------------------% % Complement of a set


%--------------------------------------%
\subsection*{ Three Sets }

%- Section 2.3.5 %- Associative Law %- Distribution Law





%-------------------------------------% %- Section 3
Propositional Logic A statement is a declarative sentence that
is either true or false.
\begin{itemize}
\item $\tilde q$ not q \item $p \vee q$ \item $p \wedge \tilde
q$
\end{itemize}



%======================================================================================= %

Question 5

-----------------------------------------------------


--------------------------------------------

Let A, B be subsets of the universal set \mathcal{U}.

Use membership tables to prove De Morgan's Laws.



Construct Membership tables for each of the sets
(A-B) - C
A-(B- C)

(A-B) -C = A-(B-C)
A



\begin{itemize}
\item[a.] (1 mark) Write out the sample space for the outcomes for both players A and B.
\item[b.] (1 mark) Write out the sample space for the outcomes of C, where C is the difference of the two scores (i.e. B-A)
\item[c.] (1 mark) Are the sample points for the sample space of C equally probable? Provide a brief justification for your answer.
\end{itemize}

%----------------------------------------------------------%
\newpage
\section*{Section B: Set Operations}
\begin{itemize}
\item[B.1] complement of A $A^{\prime}$
\item[B.2] Union $A \cup B$
\item[B.3] Intersection $A \cap B$
\item[B.4] Relative Difference $A \otimes B$
\item[A.5]
\item[A.6]
\item[A.7]
\item[A.8]
\end{itemize}
\newpage


\begin{itemize}
\item Specifying Sets
\item Listing Method
\item Rules of Inclusion method
\end{itemize}


\begin{itemize}
\item Subsets Notation of a subset
\item Cardinality of a set
\item Power of a set
\end{itemize}

\subsection*{Operation on Sets}

\begin{itemize}
\item The complement of Set
\item Binary Operations
\begin{itemize}
\item Union
\item Intersection
\end{itemize}
\item Membership tables
\item Laws for Combining Sets
\end{itemize}

\newpage


\subsection*{Associative Laws}
\[ (A \cup B) \cup C =  A \cup (B \cup C)  \]
\[ (A \cap B) \cap C =  A \cap (B \cap C)  \]

\subsection*{Distributive Laws}
\[ (A \cup B) \cap C =  (A \cup B) \cap (A \cup C)  \]
\[ (A \cap B) \cup C =  (A \cap B) \cup (A \cap C)  \]


\[ (A \cup B) \cap B^{\prime} \]
%----------------------------------------------------------%
\section*{Section C: Real and Rational Numbers}
\begin{itemize}
\item[C.1]
\item[C.2]
\item[C.3]
\item[C.4]
\item[C.5]
\item[C.6]
\item[C.7]
\item[C.8]
\end{itemize}
\newpage
%----------------------------------------------------------%
\newpage
\section*{Formulae}
\begin{itemize}




\item Bayes' Theorem:
\begin{equation*}
P(B|A)=\frac{P\left(A|B\right) \times P(B) }{P\left( A\right) }.
\end{equation*}



\item Binomial probability distribution:
\begin{equation*}
P(X = k) = ^{n}C_{k} \times p^{k} \times \left( 1-p\right) ^{n-k}\qquad \left( \text{where}\qquad
^{n}C_{k} =\frac{n!}{k!\left(n-k\right) !}. \right)
\end{equation*}

\item Poisson probability distribution:
\begin{equation*}
P(X = k) =\frac{m^{k}\mathrm{e}^{-m}}{k!}.
\end{equation*}
\end{itemize}



%-------------------------------------%

{Set Theory : Venn Diagrams}

%%- \vspace{-0.5cm}
\begin{itemize}
\item Let $A$,$B$ and $C$ be subsets of a universal set $U$.
\item Draw a labelled Venn diagram depicting $A$,$B$,$C$ in such a way that they divide $U$ into 8 disjoint regions.
\end{itemize}


%-------------------------------------%

{Subsets and Elements of Sets}

%%- \vspace{-1cm}
\begin{center}
\begin{tabular}{|c|c|c|c|c|}
\hline Region & A & B & C &  \\ 
\hline 1 &\phantom{sp} No \phantom{sp}&\phantom{sp} No\phantom{sp} & \phantom{sp} No \phantom{sp} &  \\ 
\hline 2 & Yes & No & No &  \\ 
\hline 3 & No & Yes & No &  \\ 
\hline 4 & Yes & Yes & No &  \\ 
\hline 5 & No & No & Yes &  \\ 
\hline 6 & Yes & No & Yes &  \\ 
\hline 7 & No & Yes & Yes &  \\ 
\hline 8 & Yes & Yes & Yes & $A \cap B \cap C$ \\ 
\hline 
\end{tabular}
\end{center} 

%-------------------------------------%

{Subsets and Elements of Sets}

%%- \vspace{-1cm}
\begin{center}
\begin{tabular}{|c|c|c|c|c|}
\hline Region & A & B & C &  \\ 
\hline 1 &\phantom{sp} No \phantom{sp}&\phantom{sp} No\phantom{sp} & \phantom{sp} No \phantom{sp} &  \\ 
\hline 2 & Yes & No & No &  \\ 
\hline 3 & No & Yes & No &  \\ 
\hline 4 & Yes & Yes & No &  \\ 
\hline 5 & No & No & Yes &  \\ 
\hline 6 & Yes & No & Yes &  \\ 
\hline 7 & No & Yes & Yes &  \\ 
\hline 8 & Yes & Yes & Yes & $A \cap B \cap C$ \\ 
\hline 
\end{tabular}
\end{center} 

%-------------------------------------%

{Subsets and Elements of Sets}

%%- \vspace{-1cm}
\begin{center}
\begin{tabular}{|c|c|c|c|c|}
\hline Region & A & B & C &  \\ 
\hline 1 &\phantom{s} No \phantom{s}&\phantom{s} No\phantom{s} & \phantom{s} No \phantom{s} & $(A \cup B \cup C)^C$  \\ 
\hline 2 & Yes & No & No &  \\ 
\hline 3 & No & Yes & No &  $B-(A \cap C)$\\ 
\hline 4 & Yes & Yes & No & $(A \cap B) - C$ \\ 
\hline 5 & No & No & Yes &  $C-(A \cap B)$ \\ 
\hline 6 & Yes & No & Yes & $(A \cap C) - B$ \\ 
\hline 7 & No & Yes & Yes & $(B \cap C) - A$ \\ 
\hline 8 & Yes & Yes & Yes & $A \cap B \cap C$ \\ 
\hline 
\end{tabular}
\end{center} 


%-------------------------------------%

{Subsets and Elements of Sets}

%%- \vspace{-0.5cm}
\textbf{Elements of a Set}\\
\begin{itemize}
\item Sets are comprised of members, which are often called \textbf{elements}. 
\item If a particular value ($k$) is an element of set $A$, then we would write this as
\[k \in A \]

\item If a single value $k$ is not an element of set $A$, then we write
\[k \notin A \]
\end{itemize}

%------------------------------------- %

{Subsets and Elements of Sets}

%%- \vspace{-1.5cm}
\textbf{Subsets}\\
Given two sets $A$ and $B$, the set $A$ is a \textbf{subset} of set $B$ if every element of $A$ is also an element of $B$. 


We write this mathematically as
\[A \subseteq B \]


\bigskip
Sets are denoted with curly braces, even if they contain only one element.


%------------------------------------- %

{Subsets and Elements of Sets}

%%- \vspace{-0.4cm}
\textbf{Subsets}\\
Suppose we have the set $A$ comprised of the following elements
\[ A =\{3,5,7,9\}\]
The value $5$ is an element of $A$
\[  5 \in A \]

The single element set $\{5\} $ is a subset of $A$.
\[ \{5\} \subseteq A\]


%-------------------------------------%

{Subsets and Elements of Sets}

%%- \vspace{-2cm}
\textbf{Proper Subsets}\\
Given two sets $A$ and $B$, the set $A$ is a \textbf{proper subset} of set $B$ if every element of $A$ is also an element of $B$, but there are elements of set $B$ that are not in set $A$.


We write this mathematically as
\[A \subset B \]


%-------------------------------------%

{Subsets and Elements of Sets}

%%- \vspace{-2cm}
\textbf{Equivalent Sets}\\
If both of the following two statements are \textbf{true}, 
\[\mbox{1)  } A \subseteq B \]
\[\mbox{2)  } B \subseteq A \]

then $A$ and $B$ are \textbf{equivalent sets}.





%-------------------------------------%

{Subsets and Elements of Sets}

%%- \vspace{-2cm}
\textbf{Non-Comparable Sets}\\
If both of the following two statements are \textbf{false}, 
\[\mbox{1)  } A \subseteq B \]
\[\mbox{2)  } B \subseteq A \]

then $A$ and $B$ are said to be said to be \textbf{noncomparable sets}.




%-------------------------------------%

\subsection{Subsets and Elements of Sets}
%%- \vspace{-0.5cm}
\textbf{Elements of a Set}\\



\newpage

(a) A sequence is given by the recurrence relation
\[un+1 = un + n \]and u1 = 0.
% (i) Calculate u3, u4, and u5. [2]
(ii) Use induction to prove that
un =
n(n - 1)
2
forall n = 1.
[5]

%=======================================================================%
) Another sequence is defined by the recurrence relation un = un-1+ 2n-1 and
u1 = 1.
% (i) Calculate u2, u3, u4 and u5.
(ii) Prove by induction that un = n
2
for all n = 1.
\end{document} 
