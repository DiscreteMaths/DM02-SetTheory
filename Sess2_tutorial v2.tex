

% \item The Fibonacci sequence $f_n$ is defined recursively by the rule
%   \begin{equation*}
%     \begin{cases}
%       f_0&=0\\
%       f_1&=1\\
%       f_n&=a_{n-1}+a_{n-2}
%     \end{cases}
%   \end{equation*}

%   \begin{enumerate}
%   \item
%     Write a program to evaluate the Fibonacci sequence and hence evaluate $f_{50}$.
%   \item
%     Let the sequence $g_n$ be defined as the ratio
%     \begin{equation*}
%       g_n = \frac{f_{n+1}}{f_n}
%     \end{equation*}
%     Write a program to evaluate the first $50$ terms of the sequence $g_n$.
%   \item
%     Assumming that the sequence $g_n$ has a limit $\phi$, find this limit.
%   \end{enumerate}


%--------------------------------- %
\section*{The Universal Set and the Empty Set}
\begin{itemize}
\item The first is the \textbf{\textit{universal set}}, typically denoted $U$. This set is all of the elements that we may choose from. This set may be different from one setting to the next. 

\item For example one universal set may be the set of all real numbers, denoted $\mathbb{R}$, whereas for another problem the universal set may be the whole numbers $\{0, 1, 2,\ldots\}$.

\item The other set that requires consideration is called the \textit{\textbf{empty set}}. The empty set is the unique set is the set with no elements. We write this as $\{ \}$ and denote this set by $\emptyset$.
\end{itemize}



\newpage

\subsection*{Exercises}
For each of the following sets, write out the set using the listing method.
Also write down the cardinality of each set.

\begin{itemize} 
\item $\{ s : s $ is an negative integer $ -10 \leq s \leq 0 \}$
\item $\{ t : t $ is an even number $ 1 \leq t \leq 20 \}$
\item $\{ u : u $ is a prime number $ 1 \leq u \leq 20 \}$
\item $\{ v : v $ is a multiple of 3 $ 1 \leq v \leq 20 \}$
\end{itemize}
%-------------------------------------------------% 
\newpage
\section*{Power Sets}
\subsection*{Worked Example}
Consider the set $Z$:
\[ Z = \{ a,b,c\}  \]
\begin{itemize}
\item[(i)] How many sets are in the power set of $Z$? 
\item[(ii)] Write out the power set of $Z$. 
\item[(iii)] How many elements are in each element set?
\end{itemize}
%----------------------------------------------%
\subsection*{Solutions to Worked Example}

\begin{itemize}


\item[(i)] There are 3 elements in $Z$. So there is $2^3 = 8$ element sets contained in the power set.

\item[(ii)] Write out the power set of $Z$.
\[ \mathcal{P}(Z) = \{ \emptyset, \{a\}, \{b\}, \{c\}, \{a,b\}, \{a,c\}, \{b,c\}, \{a,b,c\} \} \]

\item[(iii)]
\begin{itemize}
\item[$\bullet$] One element set is the null set - i.e. containing no
elements \item[$\bullet$] Three element sets have only elements \item[$\bullet$]
Three element sets have two elements \item[$\bullet$] One element set
contains all three elements \item[$\bullet$] 1+3+3+1=8
\end{itemize}
\end{itemize}
\subsection*{Exercise}
For the set $Y = \{u,v,w,x\}$ , answer the questions from the
previous exercise


%------------------------------------------------------%

\section*{Complement of a Set}
%(2.3.1) 
Consider the universal set $U$ such that
\[U=\{2,4,6,8,10,12,15\} \]
For each of the sets $A$,$B$,$C$ and $D$, specify the complement sets.
{

\begin{center}
\begin{tabular}{|c|c|}
  \hline
Set &\phantom{sp} Complement \phantom{sp}\\
\hline \phantom{sp} $A=\{4,6,12,15\}$ \phantom{sp} &
$A^{\prime}=\{2,8,10\}$ \\ \hline $B=\{4,8,10,15\}$ & \\ \hline
$C=\{2,6,12,15\}$ & \\ \hline $D=\{8,10,15\}$ & \\ \hline

\end{tabular}
\end{center}
}
%-------------------------------------%
 % Binary Operations on Sets (2.3.2)
 % Union , Intersection, Symmetric Difference
 % Set Difference




%=======================+============================================================== %
\subsection*{Question 2}

\begin{itemize}
\item $\{12,13,14,15,16,17\}$
\item $\{0,5,-5,10,-10,15,-15,.....\}$
\item $\{6,8,10,12,14,16,18\}$
\end{itemize}
\subsection*{Question 8}
\begin{itemize}

\item[(i)] 

\item[(ii)]

\item[(iii)]

\end{itemize}
%--------------------------------------%
\subsection*{ Three Sets }

%- Section 2.3.5 %- Associative Law %- Distribution Law





%-------------------------------------% %- Section 3
Propositional Logic A statement is a declarative sentence that
is either true or false.
\begin{itemize}
\item $\tilde q$ not q \item $p \vee q$ \item $p \wedge \tilde
q$
\end{itemize}



%======================================================================================= %

Question 5

-----------------------------------------------------


--------------------------------------------

Let A, B be subsets of the universal set \mathcal{U}.

Use membership tables to prove De Morgan's Laws.



Construct Membership tables for each of the sets
(A-B) - C
A-(B- C)

(A-B) -C = A-(B-C)
A



\begin{itemize}
\item[a.] (1 mark) Write out the sample space for the outcomes for both players A and B.
\item[b.] (1 mark) Write out the sample space for the outcomes of C, where C is the difference of the two scores (i.e. B-A)
\item[c.] (1 mark) Are the sample points for the sample space of C equally probable? Provide a brief justification for your answer.
\end{itemize}

%----------------------------------------------------------%
\newpage
\section*{Section B: Set Operations}
\begin{itemize}
\item[B.1] complement of A $A^{\prime}$
\item[B.2] Union $A \cup B$
\item[B.3] Intersection $A \cap B$
\item[B.4] Relative Difference $A \otimes B$
\item[A.5]
\item[A.6]
\item[A.7]
\item[A.8]
\end{itemize}
\newpage


\begin{itemize}
\item Specifying Sets
\item Listing Method
\item Rules of Inclusion method
\end{itemize}


\begin{itemize}
\item Subsets Notation of a subset
\item Cardinality of a set
\item Power of a set
\end{itemize}

\subsection*{Operation on Sets}

\begin{itemize}
\item The complement of Set
\item Binary Operations
\begin{itemize}
\item Union
\item Intersection
\end{itemize}
\item Membership tables
\item Laws for Combining Sets
\end{itemize}

\newpage


\subsection*{Associative Laws}
\[ (A \cup B) \cup C =  A \cup (B \cup C)  \]
\[ (A \cap B) \cap C =  A \cap (B \cap C)  \]

\subsection*{Distributive Laws}
\[ (A \cup B) \cap C =  (A \cup B) \cap (A \cup C)  \]
\[ (A \cap B) \cup C =  (A \cap B) \cup (A \cap C)  \]


\[ (A \cup B) \cap B^{\prime} \]

%----------------------------------------------------------%
\newpage
{Set Theory : Venn Diagrams}

%%- \vspace{-0.5cm}
\begin{itemize}
\item Let $A$,$B$ and $C$ be subsets of a universal set $U$.
\item Draw a labelled Venn diagram depicting $A$,$B$,$C$ in such a way that they divide $U$ into 8 disjoint regions.
\end{itemize}


%-------------------------------------%

{Subsets and Elements of Sets}

%%- \vspace{-1cm}
\begin{center}
\begin{tabular}{|c|c|c|c|c|}
\hline Region & A & B & C &  \\ 
\hline 1 &\phantom{sp} No \phantom{sp}&\phantom{sp} No\phantom{sp} & \phantom{sp} No \phantom{sp} &  \\ 
\hline 2 & Yes & No & No &  \\ 
\hline 3 & No & Yes & No &  \\ 
\hline 4 & Yes & Yes & No &  \\ 
\hline 5 & No & No & Yes &  \\ 
\hline 6 & Yes & No & Yes &  \\ 
\hline 7 & No & Yes & Yes &  \\ 
\hline 8 & Yes & Yes & Yes & $A \cap B \cap C$ \\ 
\hline 
\end{tabular}
\end{center} 

%-------------------------------------%

{Subsets and Elements of Sets}

%%- \vspace{-1cm}
\begin{center}
\begin{tabular}{|c|c|c|c|c|}
\hline Region & A & B & C &  \\ 
\hline 1 &\phantom{sp} No \phantom{sp}&\phantom{sp} No\phantom{sp} & \phantom{sp} No \phantom{sp} &  \\ 
\hline 2 & Yes & No & No &  \\ 
\hline 3 & No & Yes & No &  \\ 
\hline 4 & Yes & Yes & No &  \\ 
\hline 5 & No & No & Yes &  \\ 
\hline 6 & Yes & No & Yes &  \\ 
\hline 7 & No & Yes & Yes &  \\ 
\hline 8 & Yes & Yes & Yes & $A \cap B \cap C$ \\ 
\hline 
\end{tabular}
\end{center} 

%-------------------------------------%

{Subsets and Elements of Sets}

%%- \vspace{-1cm}
\begin{center}
\begin{tabular}{|c|c|c|c|c|}
\hline Region & A & B & C &  \\ 
\hline 1 &\phantom{s} No \phantom{s}&\phantom{s} No\phantom{s} & \phantom{s} No \phantom{s} & $(A \cup B \cup C)^C$  \\ 
\hline 2 & Yes & No & No &  \\ 
\hline 3 & No & Yes & No &  $B-(A \cap C)$\\ 
\hline 4 & Yes & Yes & No & $(A \cap B) - C$ \\ 
\hline 5 & No & No & Yes &  $C-(A \cap B)$ \\ 
\hline 6 & Yes & No & Yes & $(A \cap C) - B$ \\ 
\hline 7 & No & Yes & Yes & $(B \cap C) - A$ \\ 
\hline 8 & Yes & Yes & Yes & $A \cap B \cap C$ \\ 
\hline 
\end{tabular}
\end{center} 



%------------------------------------- %

{Subsets and Elements of Sets}

%%- \vspace{-1.5cm}
\textbf{Subsets}\\
Given two sets $A$ and $B$, the set $A$ is a \textbf{subset} of set $B$ if every element of $A$ is also an element of $B$. 


We write this mathematically as
\[A \subseteq B \]


\bigskip
Sets are denoted with curly braces, even if they contain only one element.


%------------------------------------- %

{Subsets and Elements of Sets}

%%- \vspace{-0.4cm}
\textbf{Subsets}\\
Suppose we have the set $A$ comprised of the following elements
\[ A =\{3,5,7,9\}\]
The value $5$ is an element of $A$
\[  5 \in A \]

The single element set $\{5\} $ is a subset of $A$.
\[ \{5\} \subseteq A\]


%-------------------------------------%

{Subsets and Elements of Sets}

%%- 


%-------------------------------------%

{Subsets and Elements of Sets}

%%- 
\textbf{Equivalent Sets}\\
If both of the following two statements are \textbf{true}, 
\[\mbox{1)  } A \subseteq B \]
\[\mbox{2)  } B \subseteq A \]

then $A$ and $B$ are \textbf{equivalent sets}.





%-------------------------------------%

{Subsets and Elements of Sets}

%%- 
\textbf{Non-Comparable Sets}\\
If both of the following two statements are \textbf{false}, 
\[\mbox{1)  } A \subseteq B \]
\[\mbox{2)  } B \subseteq A \]

then $A$ and $B$ are said to be said to be \textbf{noncomparable sets}.




%-------------------------------------%

\subsection{Subsets and Elements of Sets}
%%- \vspace{-0.5cm}
\textbf{Elements of a Set}\\



\newpage


